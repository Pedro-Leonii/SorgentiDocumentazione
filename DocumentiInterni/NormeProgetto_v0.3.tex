\documentclass[a4paper, 12pt]{article}

\usepackage[italian]{babel}
\usepackage{tikz}
\usepackage{xcolor}
\usepackage{graphicx}
\usepackage{hyperref}
\usepackage{imakeidx}
\usepackage{caption}
\usepackage{fancyhdr}
\usepackage{tabularx}
\usepackage{ragged2e}

%--------------------VARIABILI--------------------
\def\logo{../Immagini/logo.jpeg}
\def\ultima-versione{v0.3}
\def\titolo{Norme di Progetto }
%------------------------------------------------

\usetikzlibrary{calc}
\definecolor{fp-blue}{HTML}{2885c8}
\definecolor{fp-red}{HTML}{ea5f64}
\makeindex[title=Indice]
\hypersetup{hidelinks}

\pagestyle{fancy}
\fancyhead[L]{}
\setlength{\headheight}{15pt}
\fancyhead[R]{\titolo - \ultima-versione}

\renewcommand{\familydefault}{\sfdefault}
\newcommand{\glossario}[1]{\fontfamily{lmr}\selectfont{\textit{#1\textsubscript{\small G}}}}

%--------------------INFORMAZIONI PRIMA PAGINA-------------------- 
\title{\Huge \textbf{\titolo}}
\author{\Large{Alt} \raisebox{0.3ex}{\normalsize  +} \Large{F4}}
\date{9 novembre 2024}
%----------------------------------------------------------------

\begin{document}

\begin{titlepage}      
    \maketitle
    \thispagestyle{empty}  

    \begin{tikzpicture}[remember picture, overlay]
        \fill[fp-blue] 
        ($(current page.south west) + (0, 10)$) 
        -- ($(current page.center) + (0, -8)$)
        -- ($(current page.center) + (0, -15)$)
        -- (current page.south west);

        \fill[fp-red]
        ($(current page.south east) + (0, 10)$) 
        -- ($(current page.center) + (0, -8)$)
        -- ($(current page.center) + (0, -15)$)
        -- (current page.south east);

        \clip ($(current page.center) + (0, -8)$) circle (1cm) node 
        {\includegraphics[width=.25\textwidth]{\logo}};
        
    \end{tikzpicture}    
\end{titlepage}

\newpage

\begin{table}[!h]
    \centering
    \caption*{\textbf{\Large Registro Modifiche}}
    {\renewcommand{\arraystretch}{2}
    \begin{tabularx}{\textwidth}{| X | X | X | X |}
        \hline
            \textbf{\large Versione} & 
            \textbf{\large Data} & 
            \textbf{\large Autore/i} & 
            \textbf{\large Descrizione} \\ 
        \hline \hline
        \ultima-versione & 
            09/11/2024 & 
            Francesco Savio & 
            Modifica tipi di documento  \\
        \hline 
            v0.2 & 
            26/10/2024 & 
            Pedro Leoni & 
            Modifica sezioni "template" e "organizzazione repository"  \\
        \hline 
            v0.1 & 
            25/10/2024 & 
            Enrico Bianchi & 
            Stesura documento \\
        \hline 
    \end{tabularx}}
\end{table}

\newpage

\tableofcontents

\newpage

    \section{Introduzione}
    Il seguente documento "Norme di Progetto" ha lo scopo di definire strumenti di lavoro e pratiche comuni adottati dal team per garantire una metodologia di lavoro efficiente ed efficace.
    Questo documento verrà quindi utilizzato come guida dal team in caso vi siano dubbi su come lavorare.

    \section{Canali di comunicazione}
    
    \subsection{Comunicazioni interne}
    Il gruppo ha stabilito di utilizzare Telegram come chat di gruppo per le discussioni rapide e di usare Discord per le riunioni. 
    Il motivo di questa scelta è di mantenere una comunicazione fluida e immediata tra i membri del team, facilitando il coordinamento e la condivisione di idee.
    
    \subsection{Comunicazioni esterne}
    Si è deciso di utilizzare Gmail come servizio di posta elettronica per comunicare sia con il proponente che con il docente. 
    Per le riunioni con il proponente, si opterà per Zoom, che offre funzionalità avanzate per le conferenze online.

    \section{Documentazione}
    
    \subsection{Tipi di documenti}
    \subsubsection{verbale interno}
    \begin{itemize}
        \item nome del file: aaaa\_mm\_dd-vX\_Y.tex
        \item template: usare il file varbale\_interno.tex presente dentro la cartella template
        \item versionamento: settare la variabile ultima-versione a v0.1 se il file è appena creato o aumentarla di +0.1 rispetto alla versione corrente, solo nel caso di approvazione va di default a v1.0
        \newline Aggiornare il registro delle modifiche aggiungendo una nuova riga della tabella in cima, togliere la variabile ultima-versione presente e mettere il valore manualmente, usare la variabile ultima-versione nella nuova riga creata al top della tabella
        \newline Aggiornare il nome del file con la nuova versione
        \item cartella di destinazione: Candidatura \texttt{->} VerbaliInterni
    \end{itemize}
    \subsubsection{verbale esterno}
    \begin{itemize}
        \item nome del file: nomeAzienda-aaaa\_mm\_dd-vX\_Y.tex
        \item template: usare il file varbale\_esterno.tex presente dentro la cartella template
        \item versionamento: settare la variabile ultima-versione a v0.1 se il file è appena creato o aumentarla di +0.1 rispetto alla versione corrente, solo nel caso di approvazione va di default a v1.0
        \newline Aggiornare il registro delle modifiche aggiungendo una nuova riga della tabella in cima, togliere la variabile ultima-versione presente e mettere il valore manualmente, usare la variabile ultima-versione nella nuova riga creata al top della tabella
        \newline Aggiornare il nome del file con la nuova versione
        \item cartella di destinazione: Candidatura \texttt{->} VerbaliEsterni
    \end{itemize}
    \subsubsection{documento generico}
    \begin{itemize}
        \item nome del file\: nomeFile\-vX\_Y.tex (il nome se composto deve avere \_ come separatore\, nel caso non debba essere versionato togliere il \-v)
        \item template: usare il file generico.tex presente dentro la cartella template
        \item versionamento: se il file non ha versione rimuovere la variabile ultima-versione e la tabella delle modifiche, se dev'essere versionato procedere con gli stessi passaggi di verbale esterno.
        \item cartella di destinazione: non specificata
    \end{itemize}
    
    

\end{document}