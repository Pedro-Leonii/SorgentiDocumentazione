\documentclass[a4paper, 12pt]{article}

\usepackage[italian]{babel}
\usepackage{tikz}
\usepackage{xcolor}
\usepackage{graphicx}
\usepackage{hyperref}
\usepackage{imakeidx}
\usepackage{caption}
\usepackage{fancyhdr}
\usepackage{geometry}
\usepackage{tabularx}
\usepackage{listings}


%--------------------VARIABILI--------------------
\def\logo{../Immagini/logo.jpeg}
\def\ultima-versione{ ULTIMA VERSIONE }
\def\titolo{ TITOLO DOCUMENTO }
%------------------------------------------------

\lstnewenvironment{TeXlstlisting}{\lstset{language=[LaTeX]TeX}}{}
\usetikzlibrary{calc}
\definecolor{fp-blue}{HTML}{2885c8}
\definecolor{fp-red}{HTML}{ea5f64}
\makeindex[title=Indice]
\hypersetup{
    colorlinks=true,
    linkcolor=fp-red,
    filecolor=magenta,      
    urlcolor=fp-blue,
    pdfpagemode=FullScreen,
}
\pagestyle{fancy}
\fancyhead[L]{}
\setlength{\headheight}{15pt}
\fancyhead[R]{\titolo - \ultima-versione}

\renewcommand{\familydefault}{\sfdefault}
\newcommand{\glossario}[1]{\fontfamily{lmr}\selectfont{\textit{#1\textsubscript{\small G}}}}

\newcolumntype{C}{>{\centering\arraybackslash}X}

\lstset{
    basicstyle=\ttfamily\footnotesize,             
    stepnumber=1,                                              
    tabsize=2,                        
    showspaces=false,                   
    showstringspaces=false,           
    breaklines=true,                      
    breakatwhitespace=true,                          
}



%--------------------INFORMAZIONI PRIMA PAGINA-------------------- 
\title{\Huge \textbf{\titolo}}
\author{\Large{Alt} \raisebox{0.3ex}{\normalsize  +} \Large{F4}}
\date{ 11 novembre 2024}
%----------------------------------------------------------------

\begin{document}

\begin{titlepage}      
    \maketitle
    \thispagestyle{empty}  

    \begin{tikzpicture}[remember picture, overlay]
        \fill[fp-blue] 
        ($(current page.south west) + (0, 10)$) 
        -- ($(current page.center) + (0, -8)$)
        -- ($(current page.center) + (0, -15)$)
        -- (current page.south west);

        \fill[fp-red]
        ($(current page.south east) + (0, 10)$) 
        -- ($(current page.center) + (0, -8)$)
        -- ($(current page.center) + (0, -15)$)
        -- (current page.south east);

        \clip ($(current page.center) + (0, -8)$) circle (1cm) node 
        {\includegraphics[width=.25\textwidth]{\logo}};
        
    \end{tikzpicture}    
\end{titlepage}

\thispagestyle{plain}
\newgeometry{ignoreall, hmargin=20pt}
\begin{table}[!h]
    \centering
    \caption*{\textbf{\Large Registro Modifiche}}
    {\renewcommand{\arraystretch}{2}
    \begin{tabularx}{\textwidth}{| c | c | C | C | C |}
        \hline
            \textbf{\normalsize Versione} & 
            \textbf{\normalsize Data} & 
            \textbf{\normalsize Autore/i} & 
            \textbf{\normalsize Verificatore} &
            \textbf{\normalsize Descrizione} \\ 
        \hline \hline
         & 
        8 novembre 2024  & 
        Pedro Leoni &
        & 
        Sottosezioni \hyperref[subsec:cre]{Guida a \LaTeX}, 
        \hyperref[subsec:cre]{Creazione documento}, \hyperref[subsec:cre]{Modifica documento}, \hyperref[subsec:ver]{Verifica documento}, \hyperref[subsec:acc]{Accettazione documenti}, \hyperref[subsec:vers]{Versionamento}, \hyperref[subsec:docs]{Tipi di documenti}, \hyperref[subsec:mod]{Modalità di scrittura}, \hyperref[subsec:pub]{Pubblicazione documentazione}, \hyperref[subsec:issue]{Issue}\\
        \hline 
    \end{tabularx}}
\end{table}
\restoregeometry

\tableofcontents

\newpage

\section{Documentazione}

\subsection{Guida a \LaTeX}

\subsubsection{Creazione tabella}
\label{subsub:tab}
\begin{lstlisting}

        \begin{table}[!h]
            \begin{tabularx}[| X | c | l | r |]
                \hline

                Intestazione1 &
                Intestazione2 &
                Intestazione3 &
                Intestazione4 \\

                \hline

                prima colonna prima riga   &
                seconda colonna prima riga &
                terza coolonna prima riga  &
                quarta colonna prima riga  &

                \hline

                ...

                \hline
            \end{tabularx}
            \caption{Riassunto contenuto}
        \end{table}
\end{lstlisting}
Dove:
\begin{itemize}
    \item \lstinline+[!h]+ indica che il posizionamento della tabella nel pdf deve essere dove si trova nel codice.
    
    \item \lstinline+[| X | c | l | r |]+ indica le colonne della tabella.

    \lstinline+|+ indica una riga di separazione tra le colonne.

    \lstinline+X+ indica una colonna che si allarga dinamicamente.

    \lstinline+c+, \lstinline+l+ e \lstinline+r+ indicano rispettivamente colonne in cui il posizionamento del testo è centrale, a sinistra e a destra.

    \item \lstinline+\hline+ crea una linea orizzontale.
\end{itemize}

\subsubsection{Link}
\label{subsub:link}
Link ad elementi interni alla pagina:
\begin{lstlisting}
    \label[sec:nome]

    ... 

    \hypperref[sec:nome]{nomeLink}
\end{lstlisting}
Dove:
\begin{itemize}
    \item Convezione standard per nominare le label: \lstinline|tipoElemento:nome|.
    
    Esempi: img(immagini), sec(section), subsec(subsection), subsub(subsubsection), tab(tabella) ecc.
\end{itemize}
\noindent Link ad elementi esterni alla pagina:
\begin{lstlisting}
    \href{URL}{nomeLink}
\end{lstlisting}

\subsubsection{Listati di codice}
\label{subsub:cod}
Listati di codice su più righe:
\begin{TeXlstlisting}
\begin{lstlisting}
    ...
\end{lstlisting}
\end{TeXlstlisting}
\noindent 
Listati di codice inline:
\begin{lstlisting}
    \lstinline|code|
    \lstinline+code+
\end{lstlisting}

\subsubsection{Figure}
\label{subsub:fig}
\begin{lstlisting}
    \begin{figure}[h!]
        \includegraphics[scale=1.2]{pathImmagine}
        \caption{Riassunto immagine}
    \end{figure}
\end{lstlisting}
Dove:
\begin{itemize}
    \item \lstinline|[h!]| permette di posizionare la figura nel pdf dove si trova nel codice.
    \item \lstinline|scale=x.y| indica la scala da applicare all'immagine.
\end{itemize}

\subsection{Creazione documento}
\label{subsec:cre}
Un documento viene creato solo se esiste una issue che lo prevede.
La creazione di un documento segue i passi:
\begin{enumerate}
    \item Presa in carico della issue all'interno della board-view del \href{https://github.com/orgs/ALT-F4-eng/projects}{project GitHub}.
    Questo richiede:
    \begin{enumerate}
        \item Auto assegnamento della issue.
        \item Spostamento della issue nella colonna "In Progress" della board view dell'iterazione corrente.
    \end{enumerate}

    \item Spostarsi sul branch develop tramite il comando:
    \begin{lstlisting}
        git checkout develop
    \end{lstlisting}
     
    \item Aggiornare il branch develop tramite il comando: 
    \begin{lstlisting}
        git pull origin develop
    \end{lstlisting}
    
    \item Creazione e spostamento nel branch \lstinline|feature/#id-issue| tramite il comando:
    \begin{lstlisting}
        git checkout -b feature/#id-issue
    \end{lstlisting}

    \item Creazione del documento usando template e nomi(\underline{esclusa versione}) indicati nella sezione \hyperref[subsec:docs]{Tipi di documenti}.
    
    Per i documenti di tipo: Verbali interni e Verbali esterni indicare la data in cui sono stati svolti mentre per gli altri documenti indicare la data attuale.
    
    \item Popolazione del documento assicurandosi di soddisfare la definition of done indicata nella descrizione dell'issue.
    
    La scrittura deve rispettare la \hyperref[subsec:mods]{Modalità di scrittura}. 
    
    \item Compilazione registro delle modifiche omettendo le colonne: Verificatore, Versione.
    Nella colonna Descrizione deve essere inserito un \hyperref[subsub:link]{link} agli elementi creati.
    
    \item Notare che non viene modificato il nome del documento. 
    
    \item Registrazione delle modifiche nel repository locale tramite i comandi:
    \begin{lstlisting}
        git add *.tex
        git commit -m "messaggio-commit"
    \end{lstlisting}

    \item Aggiornare il ramo develop locale e mergiarlo nel ramo di feature tramite i comandi:
    \begin{lstlisting}
        git checkout develop
        git pull origin develop

        git checkout feature/#id-feature
        git merge develop
    \end{lstlisting}

    \item Pubblicazione del ramo nel repository tramite il comando:
    \begin{lstlisting}
        git push origin feature/#id-issue
    \end{lstlisting}

    \item Creazione di una pull request nel \href{https://github.com/ALT-F4-eng/SorgentiDocumentazione}{repository del progetto} che parte dal ramo feature verso il ramo develop.
    La descrizione della pull request deve contenere il testo \lstinline|closes id-issue|.
    
    \item Aggiunta della pull request nella colonna "Backlog" della board view del progetto.
    
    \lstinline|Progetto > (Backlog) + Add item|
    
    \item Assegnazione del valore "CurrentIteration" al field "Iteration" della pull request nella table view del progetto.
    
    \lstinline|TabellaProgetto > Iteration > CurrentIteration|
   
\end{enumerate}

\subsection{Modifica documento}
\label{subsec:mod}
Un documento viene modificato solo se esiste una issue che lo prevede.
La modifica di un documento segue i passi:
\begin{enumerate}

    \item Presa in carico della issue all'interno della board view del \href{https://github.com/orgs/ALT-F4-eng/projects}{project GitHub}.
    Questo richiede:
    \begin{enumerate}
        \item Auto assegnamento della issue.
        \item Spostamento della issue nella colonna "In Progress" della board view dell'iterazione corrente.
    \end{enumerate}
    
    \item Spostarsi sul branch develop tramite il comando:
    \begin{lstlisting}
        git checkout develop
    \end{lstlisting}
     
    \item Aggiornare il branch develop tramite il comando: 
    \begin{lstlisting}
        git pull origin develop
    \end{lstlisting}
    
    \item Creazione e spostamento branch \lstinline|feature/#id-issue| tramite il comando:
    \begin{lstlisting}
        git checkout -b feature/#id-issue
    \end{lstlisting}
    

    \item Modifica del contenuto del documento assicurandosi di soddisfare la definition of done indicata nella descrizione dell'issue.
     
    La scrittura deve rispettare la \hyperref[subsec:mods]{Modalità di scrittura}. 

    \item Compilazione registro delle modifiche omettendo le colonne: Verificatore, Versione.
    Nella colonna Descrizione deve essere inserito un \hyperref[subsub:link]{link} agli elementi modificati.

    Per i documenti di tipo diverso da Verbali interni e Verbali esterni deve essere aggiornata la variabile \LaTeX \space chiamata \lstinline|\date| usando la data attuale indicando il mese in lettere. 
    
    \item Notare che non viene modificato il nome del documento.

    \item Pubblicazione del ramo nel repository tramite il comando:
    \begin{lstlisting}
        git push origin feature/#id-issue
    \end{lstlisting}

    \item Creazione di una pull request nel \href{https://github.com/ALT-F4-eng/SorgentiDocumentazione}{repository del progetto} che parte dal ramo feature verso il ramo develop.
    La descrizione della pull request deve contenere il testo \lstinline|closes #id-issue|.

    \item Aggiunta della pull request nella colonna "Backlog" della board view del progetto.
        
    \lstinline|Progetto > (Backlog) + Add item|
    
    \item Assegnazione del valore "CurrentIteration" al field "Iteration" della pull request nella table view del progetto.
    
    \lstinline|TabellaProgetto > Iteration > CurrentIteration|

\end{enumerate}

\subsection{Verifica documento}
\label{subsec:ver}
Un documento deve essere verificato a seguito della sua modifica o della sua creazione.
La verifica di un documento segue i passi:
\begin{enumerate}
    \item Presa in carico della pull request all'interno della board view del \href{https://github.com/orgs/ALT-F4-eng/projects}{project GitHub}.
    Questo richiede:
    \begin{enumerate}
        \item Auto assegnamento della review della pull request.
        \item Spostamento della pull request nella colonna "In Progress" della board view dell'iterazione corrente.
    \end{enumerate}
    
    \item Ottenimento del ramo di feature tramite i comandi:
    \begin{lstlisting}
        git checkout -b feature/#id-issue
        git pull origin feature/#id-issue
    \end{lstlisting}
    L'id della issue relativa alla pull request si trova nella descrizione della pull request stessa.

    \item Compilazione del documento e verifica del contenuto rispetto alla definition of done contenuta nella relativa issue.
    
    \item Se la definition of done non è stata rispettata commentare la pull request indicando le parti mancanti.
    
    Il verificatore deve spuntare i task della definition of done che sono stati portati a termine.
    
    \item Controllo grammaticale.
    
    \item Aggiornare il registro delle modifiche popalando le colonne: Verificatore e Versione.
    
    Il cambiamento della versione richiede i seguenti passi:
    \begin{itemize}
        \item Modifica della variabile \LaTeX \space chiamata \lstinline|\ultima-versione| seguendo la convenzione descritta nella sezione \hyperref[subsec:vers]{Versionamento}.
        Questa variabile deve essere usata nella colonna Versione del registro delle modifiche.

        \underline{Occhio alla "v" che precede la versione.}

        \item Modifica del valore della colonna Versione nella riga precedente assegnando la versione indicata nel nome del documento sorgente. 
    \end{itemize}

    \item Modifica del nome del documento assegnando la nuova versione.

    \item Registrazione delle modifiche nel repository locale tramite i comandi:
    \begin{lstlisting}
        git add *.tex
        git commit -m "messaggio-commit"
    \end{lstlisting}

    \item Allineamento con lo stato attuale del ramo develop remoto tramite i comandi:
    \begin{lstlisting}
        git checkout develop
        git pull origin develop

        git checkout feature/#id-issue
        git merge develop
    \end{lstlisting}

    \item Pubblicazione del ramo nel repository tramite il comando:
    \begin{lstlisting}
        git push origin feature/#id-issue
    \end{lstlisting}

    \item Accettazione e merge della pull request.
    
    \item Eliminazione del ramo di feature.
\end{enumerate}

\subsection{Accettazione documenti}
\label{subsec:acc}
L'accettazione dei documenti viene fatta quando è necessario pubblicare dei documenti compilati.
Questo viene fatto quando è necessario mostrare all'esterno del team delle versioni "stabili" dei documenti.
L'accettazione di un documento segue i passi:
\begin{enumerate}
    \item Spostarsi sul branch develop tramite il comando:
    \begin{lstlisting}
        git checkout develop
    \end{lstlisting}
     
    \item Aggiornare il branch develop tramite il comando: 
    \begin{lstlisting}
        git pull origin develop
    \end{lstlisting}
    
    \item Creazione e spostamento nel branch \lstinline|release/NomeMilestone| tramite il comando:
    \begin{lstlisting}
        git checkout -b release/NomeMilestone
    \end{lstlisting}

    \item Controllo dei documenti da pubblicare.
    
    \item Per ogni documento da pubblicare controllato:
    \begin{itemize}
        \item Modificare il valore della variabile \LaTeX \space chiamata \lstinline|\ultima-versione| aumentando il numero di versione "stabile" e azzerando il secondo valore(spiegato nella sezione \hyperref[subsec:vers]{Versionamento}).

        \item Assegnare alla colonna dell'ultima riga del registro delle modifiche la versione indicata nel nome del documento.
        
        \item Aggiungere una riga nel registro della versione in cui popolare tutte le colonne esclusa la colonna Verificatore.  
        
        Nella colonna Versione usare la variabile \LaTeX \space chiamata \lstinline|\ultima-versione|.
        Nella colonna Descrizione usare il testo "Approvazione documento".
        Nella colonna data scrivere la data attuale indicando il \underline{mese in lettere}.
    \end{itemize}
   
    \item Eventualmente apportare modifiche ai documenti e approvarli eseguendo i passi precedenti.
    
    \item Eseguire un commit per registrare le modifiche nel repository locale tramite i comandi:
    \begin{lstlisting}
        git add *.tex
        git commit -m "messaggio-commit"
    \end{lstlisting}

    \item Merge del ramo di realese nei rami main e develop e pubblicazione delle modifiche in questi ultimi, tramite i comandi:
    \begin{lstlisting}
        git checkout main 
        git merge release/NomeMilestone
        git push origin main

        git checkout develop
        git merge release/NomeMilestone
        git push origin develop
    \end{lstlisting}
\end{enumerate}

\subsection{Versionamento}
\label{subsec:vers}
I documenti vengono versionati seguendo lo schema:
\begin{center}
    vX.Y 
\end{center}
Dove:
\begin{enumerate}
    \item X è un intero positivo non nullo aumentato a seguito della accettazione di un documento che si pensi resti invariato.
    
    L'aumento di questo valore genera automaticamente l'annullamento del secondo valore Y.

    \item Y è un intero positivo non nullo aumentato al seguito della verifica di un documento.
\end{enumerate}

\subsection{Tipi di documenti}
\label{subsec:docs}
\subsubsection{Verbali interni}
\begin{enumerate}
    \item Denominazione: \lstinline|anno_mese_giorno-vX_Y.tex| dove anno, mese e giorno sono numerici e \lstinline|vX_Y| è la versione del documento che segue lo \hyperref[subsec:vers]{schema di versionamento}.
    
    \item Template: utilizzare il template situato nel percorso \lstinline|Template/verbale_interno.tex|.
    
    \item Contenuto: Registro presenze, Verbale(Argomenti trattati e Decisioni prese) e To Do(associazione compiti-issue).

    \item Cartella di destinazione: \lstinline|Candidatura/VerbaliInterni|.
\end{enumerate}

\subsubsection{Verbali esterni}
\begin{enumerate}
    \item Denominazione: \lstinline|nomeAzienda-anno_mese_giorno-vX_Y.tex| dove anno, mese e giorno sono numerici e \lstinline|vX_Y| è la versione del documento che segue lo \hyperref[subsec:vers]{schema di versionamento}.
    
    \item Template: utilizzare il template situato nel percorso \lstinline|Template/verbale_interno.tex|.
    
    \item Contenuto: Registro presenze, Domande e Conclusioni. 
    
    \item Cartella di destinazione: \lstinline|Candidatura/VerbaliEsterni|.
\end{enumerate}

\subsubsection{Norme Progetto}
\begin{enumerate}
    \item Denominazione: \lstinline|normeProgetto-vX_Y.tex| dove \lstinline|vX_Y| è la versione del documento che segue lo \hyperref[subsec:vers]{schema di versionamento}.
    
    \item Template: utilizzare il template situato nel percorso \lstinline|Template/generico.tex|.
    
    \item Contenuto: way of working del team.

    \item Cartella di destinazione: \lstinline|DocumentiInterni|.
\end{enumerate}

\subsection{Modalità di scrittura}
\label{subsec:mods}
\begin{itemize}
    \item Preferire frasi brevi.
    \item Utilizzo di elenchi puntati.
    \item Usare sempre \hyperref[subsub:link]{link} a \hyperref[subsub:tab]{tabelle} e \hyperref[subsub:fig]{figure}.
    \item Indicare le caption delle tabelle e delle figure.
    \item Usare i comandi \LaTeX \space relativi al \hyperref[subsub:cod]{codice} per indicare: comandi, spezzoni di script, nomi file, percorsi e nomi tecnici.
    \item Scrivere i mesi delle date in lettere.
    \item Scrivere solo per risolvere un issue.
    \item Assicurarsi sempre di soddisfare la definition of done delle issue.
\end{itemize}

\subsection{Pubblicazione documentazione}
\label{subsec:pub}
La pubblicazione della documentazione compilata avviene usando una \href{https://github.com/ALT-F4-eng/SorgentiDocumentazione/blob/main/.github/workflows/pubblicazione_compilati.yaml}{GitHub Action}.
Questa GitHub Action compila i documenti sorgenti pubblicati nel ramo main del repository \href{https://github.com/ALT-F4-eng/SorgentiDocumentazione}{SorgentiDocumentazione} e li pubblica nel ramo main del repository \href{https://github.com/ALT-F4-eng/Documentazione}{Documentaizone}. 
Per fare ciò utilizza due variabili settate a livello di repository ovvero:
\begin{enumerate}
    \item \lstinline|DIRS_TO_DEL|: contiene un insieme di nomi di cartelle che devono essere eliminate a seguito della compilazione dei documenti.
    Se si vuole evitare di pubblicare una cartella il suo nome(preceduto da spazio) deve essere aggiunto al valore di questa variabile.

    \item \lstinline|DIRS_TO_IGNORE|: contiene un insieme di nomi di cartelle che devono essere lasciate invariate nel repository in cui vengono pubblicati i file compilati.
    
    Modificando il valore di questa variabile è quindi possibile aggiungere o togliere delle cartelle da non sovrascrivere.

    Queste cartelle attualmente sono solo quelle dei verbali esterni dato che vengono firmati dal referente dell'azienda.
\end{enumerate}

\subsection{Issue}
\label{subsec:issue}
Tutto ciò che viene fatto dal team deve essere documentato da issue di GitHub.
Le issue per la documentazione devono avere una delle seguenti label:
\begin{enumerate}
    \item bug: problematiche minori riguardanti il contenuto del documento associato all'issue.
    Es: errori ortografici e di formulazione delle frasi.

    \item enhancement: automazioni che devono essere implementate.
    
    \item documentation: creazione o modifica di un documento.
    
    \item question: per approfondimenti sugli strumenti utili.
\end{enumerate}

Il titolo di ogni issue deve essere un nome parlante che permette quindi d'identificare rapidamente il compito a essa associato.

Ogni issue deve contenere nella propria descrizione la definition of done sotto forma di check list.
Questa permette al team di avere una definizione comune sullo stato in cui l'elemento riferito dalla issue soddisfa la necessità catturata dalla issue stessa.
Una check list viene definita usando il seguente codice:
\begin{lstlisting}
    - [ ] primo elemento
    - [ ] secondo elemento
    ...
\end{lstlisting}
\end{document}