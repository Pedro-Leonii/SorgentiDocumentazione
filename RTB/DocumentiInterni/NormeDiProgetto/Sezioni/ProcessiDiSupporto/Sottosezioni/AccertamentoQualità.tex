\subsection{Accertamento della Qualità}
\label{subsec:accertamento_qualità}
Il processo di accertamento della qualità ha l'obiettivo di garantire che il progetto rispetti gli standard e i requisiti definiti, sia per quanto riguarda i prodotti sviluppati che per i processi adottati. 
Questo processo rappresenta un elemento centrale nella gestione del progetto, in quanto assicura il monitoraggio continuo e la verifica della conformità alle specifiche tecniche, agli obiettivi di progetto e alle normative applicabili.
Il processo si basa su principi fondamentali quali tracciabilità, trasparenza e miglioramento continuo, e utilizza metriche e strumenti oggettivi 
per valutare sia l'efficacia dei processi che la qualità dei prodotti. 
Attraverso una pianificazione accurata e un monitoraggio costante, il processo di accertamento della qualità contribuisce a ridurre i rischi, migliorare le performance e soddisfare le aspettative degli stakeholder.
\subsubsection{Attività}
Per l'accertamento della qualità, il gruppo adotta il ciclo di Deming, o PDCA (Plan-Do-Check-Act), un approccio iterativo e continuo volto a migliorare i processi e garantire il raggiungimento degli obiettivi di qualità.
Il Ciclo di Deming è composto da 4 fasi fondamentali:
\begin{itemize}
    \item \textbf{Plan}: In questa fase, si definiscono gli obiettivi di qualità e le strategie per garantirne il raggiungimento.
    È essenziale pianificare le attività che saranno svolte durante il ciclo di vita del progetto, stabilendo le metriche da monitorare e i criteri di accettazione.
    \item \textbf{Do}: In questa fase, le attività pianificate vengono effettivamente eseguite.
    Si mettono in atto le procedure di controllo qualità stabilite per monitorare la qualità del progetto in fase di sviluppo.
    \item \textbf{Check}: Questa fase consiste nel confrontare i risultati ottenuti con gli obiettivi di qualità definiti.
    Si verificano i dati raccolti durante la fase "Do" per determinare se i processi e i risultati sono allineati con gli standard di qualità e i requisiti.
    \item \textbf{Act}: In questa fase finale, vengono intraprese azioni correttive e preventive in base ai risultati della fase "Check".
\end{itemize}
\subsubsection{Standard di riferimento per la qualità di prodotto del software}
Per la valutazione della qualità del software con conseguente stesura delle metriche viene seguita la normativa ISO/IEC 9126, secondo lo standard le qualità sono suddivise in:
\begin{itemize}
    \item \textbf{qualità esterne}: misurano i comportamenti del software durante la sua esecuzione
    \item \textbf{qualità interne}: si applicano al software non eseguibile, permettono di individuare eventuali problemi che potrebbero influire sulla qualità finale del prodotto prima che sia realizzato il software eseguibile
    \item \textbf{qualità in uso}: rappresentano il punto di vista dell'utente sul software, permettono di stabilire i seguenti obiettivi:
    \begin{itemize}
        \item \textsl{Efficacia}: capacità del software di permettere agli autenti di raggiungere gli obiettivi specificati con accuratezza e completezza
        \item \textsl{Produttività}: capcaità del software di permettere agli utenti spendere una quantità opportuna di risorse in relazione all'efficacia ottenuta
        \item \textsl{Soddisfazione}: capacità del software si soddisfare gli utenti
        \item \textsl{Sicurezza}: capacità del software di raggiungere livelli accettabili di rischi di danni a persone e apparecchiature
    \end{itemize}
\end{itemize}

Lo standard normativo stabilisce un modello definendo un set di caratteristiche che consentono di misurare e valutare diversi aspetti della qualità del prodotto software:
\begin{itemize}
    \item \textbf{Funzionalità}: capacità del software di fornire le funzioni adatte a soddisfare le esigenze stabilite
    \item \textbf{Affidabilità}: capacità del software di mantenere un livello di prestazioni specifico quando usato in date condizioni per un dato periodo di tempo
    \item \textbf{Efficienza}: capacità del software di fornire adeguate prestazioni relativamente alla quantità di risorse utilizzate
    \item \textbf{Usabilità}: capacità del software di essere propriamente compreso ed utilizzato dall'utente 
    \item \textbf{Manutenibilità}: capacità del software di essere modificato per introdurre migliorie o adattamenti
    \item \textbf{Portabilità}: capacità del software di essere di essere trasportato da un ambiente di lavoro ad un altro
\end{itemize}
\subsubsection{Notazione Metriche di Qualità}
Ogni metrica è identificata in un modo univoco seguendo la notazione: \textsl{M}.[\textsl{Tipo}].[\textsl{Abbreviazione Nome}]
Dove:
\begin{itemize}
    \item \textbf{M}: indica "metrica"
    \item \textbf{Tipo}: sarà PC per indicare che la metrica misura la qualità di un processo o PR per indicare che la metrica misura la qualità di un prodotto
    \item \textbf{Abbreviazione Nome}: verrà inserito l'acronimo basato sul nome completo della metrica  
\end{itemize}
\subsubsection{Descrizione Metriche di Qualità}
Le metriche sono descritte tramite i seguenti campi:
\begin{itemize}
    \item \textbf{Notazione}: seguendo le indicazioni sopra elencate
    \item \textbf{Nome}: nome completo della metrica 
    \item \textbf{Descrizione}: descrizione della metrica 
    \item \textbf{caratteristiche}: presente unicamente nelle metriche di prodotto, indica a quale caratteristica, tra quelle indicate nella sezione relativa allo standard di riferimento, fa riferimento la metrica
    \item \textbf{Formula di misurazione}: formula per la misurazione del valore della metrica 
\end{itemize}
Verranno poi indicati all'interno del documento "Piano di Qualifica" i valori tollerabili e i valori ottimali per ogni metrica e, nel caso delle metriche di processo, anche il processo a cui fanno riferimento.
\subsubsection{Elenco delle Metriche di Qualità} 