\subsection{UC-16}
\label{subsec:UC-16}

\begin{figure}[H]
    \includegraphics[scale=0.8]{Sezioni/UseCase/Immagini/UC-16.pdf}
    \caption{Diagramma UC-16.}
\end{figure}

\begin{usecase}{UC-16}{Esecuzione del test sul dataset corrente}

    \req{\hyperref[item:RU-5]{RU-5}} 

    \pre{
        \item Il sistema è attivo e funzionante
        \item Il dataset corrente non è vuoto
    }

    \post{
        \item L'utente conosce l'esito del test
    }
    
    \actor{Utente}

    \subactors{LLM}

    \trigger{L'utente deve testare il LLM usando il dataset corrente}
    
    \inc{\hyperref[subsec:UC-17]{UC-17}}

    \base{}

    \scenario{
        \item L'utente richiede l'esecuzione del test
        \item Per ogni coppia del dataset corrente si ottiene la risposta prodotta dal LLM sotto test
        \item Per ogni coppia del dataset corrente si calcola il grado di somiglianza tra la risposta prodotta dal LLM sotto test rispetto alla risposta attesa
        \item \texttt{<<include:UC-17>>}
    }

    \subscenario{
        \item[1.1] \textbf{Il dataset corrente contiene almeno una coppia invalida}
        \begin{itemize}
            \item[a.] hyperref[subsec:UC-11]{UC-11}
        \end{itemize}
        \item[2.1] \textbf{La comunicazione con LLM produce un errore}
        \begin{itemize}
        \item[a.] \hyperref[subsec:UC-18]{UC-18}
        \end{itemize}
    }
\end{usecase}