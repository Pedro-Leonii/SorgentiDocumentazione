\section{Use case}
\label{sec:use_case}
Gli use case sono uno strumento grafico che permette di analizzare e approfondire i requisiti utente per arrivare a definire i requisiti software del sistema.

\subsection{UC-0}

\begin{figure}[H]
    \includegraphics{Sezioni/UseCase/Immagini/UC-0.pdf}
    \caption{Diagramma UC-0.}
\end{figure}

\begin{usecase}{UC-0}{Visualizzazione del contenuto del dataset corrente}
    
    \req{\hyperref[item:RU-1]{RU-1}} 

    \pre{
        \item Il sistema è attivo e funzionante
        \item Il dataset di cui si vuole visualizzare il contenuto è stato caricato come dataset corrente
    }

    \post{
        \item Viene visualizzato il contenuto del dataset corrente
    }
    
    \actor{Utente}

    \subactors{}

    \trigger{L'utente deve visualizzare il contenuto del dataset corrente}
    
    \inc{}

    \base{}

    \scenario{
        \item L'utente richiede la visualizzazione del dataset corrente
        \item Le coppie del dataset caricato corrente vengono mostrate in una lista
    }

    \subscenario{
        \item[2.1] \textbf{Il dataset corrente è vuoto}
        \begin{itemize}
            \item[a.] Viene indicato all'utente che il dataset corrente è vuoto
        \end{itemize}
    }
\end{usecase}

\subsection{UC-1}
\begin{figure}[H]
    \includegraphics[scale=0.8]{Sezioni/UseCase/Immagini/UC-1.pdf}
    \caption{Diagramma UC-1.}
\end{figure}


\begin{usecase}{UC-1}{Inserimento di una nuova coppia nel dataset corrente}
    
    \req{\hyperref[item:RU-1]{RU-1}} 

    \pre{
        \item Il sistema è attivo e funzionante
    }

    \post{
        \item Viene inserita una nuova coppia nel dataset corrente
    }
    
    \actor{Utente}

    \subactors{}

    \trigger{L'utente vuole inserire una nuova coppia nel dataset corrente}
    
    \inc{}

    \base{}

    \scenario{
        \item L'utente richiede l'inserimento di una nuova coppia
        \item L'utente specifica il contenuto per la nuova coppia
        \item L'utente conferma l'inserimento della nuova coppia
        \item La coppia viene inserita nel dataset corrente
    }

    \subscenario{
        \item[2.1] \textbf{L'utente annulla l'operazione di inserimento} 
        \begin{itemize}
            \item[a.] \hyperref[subsec:UC-2]{UC-2}
        \end{itemize}
        \item[3.1] \textbf{La coppia da inserire non è valida}
        \begin{itemize}
            \item[a.] \hyperref[subsec:UC-3]{UC-3}
        \end{itemize}
    }
\end{usecase}




\subsection{UC-2}
\label{subsec:UC-2}
\begin{usecase}{UC-2}{Annullamento di un'operazione sul dataset corrente}

        \req{}
        
        \pre{
            \item Il sistema è attivo e funzionante     
            \item L'utente è coinvolto in un'operazione di modifica del dataset corrente
        }
        
        \post{ 
            \item L'operazione viene interrotta 
            \item Il dataset corrente resta invariato
        }

        \actor{Utente}
        
        \subactors{}
        
        \trigger{L'utente richiede l'annullamento dell'operazione}
        
        \inc{}
        
        \base{}
        
        \scenario{
            \item L'utente richiede l'annullamento dell'operazione sul dataset corrente 
            \item La modifica al dataset corrente viene annullata 
            \item Il dataset corrente resta invariato 
        }
        
        
\end{usecase}

\subsection{UC-3}
\label{subsec:UC-3}

\begin{usecase}{UC-3}{Visualizzazione errore nel contenuto di una coppia del dataset corrente}
            
    \req{}
    
    \pre{
        \item Il sistema è attivo e funzionante
        \item Sia la domanda che la risposta della coppia sono vuote
    }
    
    \post {
        \item Viene richiesta la correzione della coppia
    }
    
    \actor{Utente}
    
    \subactors{}

    \trigger{Una coppia è invalida}
    
    \inc{}
    
    \base{}
    
    \scenario{
        \item Viene visualizzato un messaggio di errore che richiede la corretta compilazione di almeno una tra domanda e risposta
    }
\end{usecase}


\subsection{UC-4}

\begin{figure}[H]
    \includegraphics{Sezioni/UseCase/Immagini/UC-4.pdf}
    \caption{Diagramma UC-4.}
\end{figure}

\begin{usecase}{UC-4}{Modifica di una coppia contenuta nel dataset corrente}
    
    \req{\hyperref[item:RU-1]{RU-1}} 

    \pre{
        \item Il sistema è attivo e funzionante
        \item Il dataset da modificare è stato caricato come dataset corrente
        \item La coppia da modificare è contenuta nel dataset corrente
    }

    \post{
        \item La coppia da modificare nel dataset corrente viene aggiornata
    }
    
    \actor{Utente}

    \subactors{}

    \trigger{L'utente deve modificare una coppia contenuta nel dataset corrente}
    
    \inc{}

    \base{}

    \scenario{
        \item L'utente richiede la modifica di una coppia del dataset corrente
        \item L'utente modifica il contenuto della coppia
        \item L'utente conferma la modifica della coppia
        \item La coppia viene aggiornata nel dataset corrente

    }

    \subscenario{
        \item[2.1] \textbf{L'utente annulla la modifica della coppia} 
        \begin{itemize}
            \item[a.] \hyperref[subsec:UC-2]{UC-2}
        \end{itemize}
        \item[3.1] \textbf{L'utente richiede di registrare una modifica invalida}
        \begin{itemize}
            \item[a.] \hyperref[subsec:UC-3]{UC-3}
        \end{itemize}
    }
\end{usecase}


\subsection{UC-5}

\begin{figure}[H]
    \includegraphics{Sezioni/UseCase/Immagini/UC-5.pdf}
    \caption{Diagramma UC-5.}
\end{figure}

\begin{usecase}{UC-5}{Eliminazione di una coppia contenuta nel dataset corrente}
    
    \req{\hyperref[item:RU-1]{RU-1}} 

    \pre{
        \item Il sistema è attivo e funzionante
        \item Il dataset corrente non è vuoto
        \item La coppia da eliminare esiste
    }

    \post{
        \item La coppia viene eliminata dal dataset corrente
    }
    
    \actor{Utente}

    \subactors{}

    \trigger{L'utente deve eliminare una coppia contenuta nel dataset corrente}
    
    \inc{}

    \base{}

    \scenario{
        \item L'utente richiede l'eliminazione della coppia
        \item L'utente conferma l'eliminazione
        \item La coppia viene eliminata dal dataset corrente
    }

    \subscenario{
        \item[2.1] \textbf{L'utente annulla l'eliminazione della coppia:} 
        \begin{itemize}
            \item[a.] \hyperref[subsec:UC-2]{UC-2}
        \end{itemize}
    }
\end{usecase}


\subsection{UC-6}

\begin{figure}[H]
    \includegraphics{Sezioni/UseCase/Immagini/UC-6.pdf}
    \caption{Diagramma UC-6.}
\end{figure}

\begin{usecase}{UC-6}{Ricerca tramite parole chiave di un sottoinsieme del dataset corrente}
    
    \req{\hyperref[item:RU-2]{RU-2}} 

    \pre{
        \item Il sistema è attivo e funzionante
        \item Il dataset corrente non è vuoto
    }

    \post{
        \item Viene mostrato il sottoinsieme del dataset corrente risultante dall'operazione di ricerca
    }
    
    \actor{Utente}

    \subactors{}

    \trigger{L'utente deve cercare una o più coppie contenute nel dataset corrente}
    
    \inc{}

    \base{}

    \scenario{
        \item L'utente specifica le parole chiave
        \item L'utente conferma l'esecuzione della ricerca
        \item Viene mostrata la lista di coppie che contengono le parole chiave nella domanda e/o nella risposta
    }

    \subscenario{
        \item[1.1] \textbf{Non vengono indicate parole chiave}
        \begin{itemize}
            \item [a.] L'utente conferma l'esecuzione della ricerca
            \item [b.] Viene mostrato l'intero dataset corrente
        \end{itemize}
        \item[3.1]\textbf{Il sottoinsieme risultante è vuoto}
        \begin{itemize}
            \item[a.] Viene notificato all'utente che la ricerca non ha prodotto risultati
        \end{itemize}
    }
\end{usecase}


\subsection{UC-7}

\begin{figure}[H]
    \includegraphics{Sezioni/UseCase/Immagini/UC-7.pdf}
    \caption{Diagramma UC-7.}
\end{figure}

\begin{usecase}{UC-7}{Modifica del nome del dataset corrente}
    
    \req{\hyperref[item:RU-1]{RU-1}} 

    \pre{
        \item Il sistema è attivo e funzionante
        \item Il dataset corrente non è vuoto 
    }

    \post{
        \item Il dataset corrente viene associato al nome indicato
    }
    
    \actor{Utente}

    \subactors{}

    \trigger{L'utente vuole modificare il nome del dataset corrente}
    
    \inc{}

    \base{}

    \scenario{
        \item L'utente richiede la modifica del nome del dataset corrente
        \item L'utente specifica il nuovo nome
        \item L'utente conferma la modifica 
        \item Il nuovo nome viene associato al dataset corrente
    }

    \subscenario{
        \item[2.1] \textbf{L'utente annulla l'operazione di modifica}
        \begin{itemize}
            \item [a.] \hyperref[subsec:UC-2]{UC-2}
        \end{itemize}
        \item[3.1] \textbf{Il nuovo nome del dataset è vuoto}
        \begin{itemize}
            \item [a.] \hyperref[subsec:UC-8]{UC-8}
        \end{itemize}
    }
\end{usecase}

\subsection{UC-8}
\label{subsec:UC-8}

\begin{usecase}{UC-8}{Visualizzazione errore nel nome di un dataset}
    
    \req{} 

    \pre{
        \item Il sistema è attivo e funzionante
        \item Il nome del dataset è invalido
    }

    \post{
        \item Viene richiesto all'utente di fornire un nome valido
    }
    
    \actor{Utente}

    \subactors{}

    \trigger{Il nome del dataset è invalido}
    
    \inc{}

    \base{}

    \scenario{
        \item Viene visualizzato un messaggio di errore che richiede
        la corretta compilazione del nome del dataset
    }
\end{usecase}

\subsection{UC-9}
\label{subsec:UC-9}


\begin{figure}[H]
    \includegraphics[scale=0.75]{Sezioni/UseCase/Immagini/UC-9.pdf}
    \caption{Diagramma UC-9.}
\end{figure}

\begin{usecase}{UC-9}{Archiviazione del dataset corrente}
    
    \req{\hyperref[item:RU-3]{RU-3}} 

    \pre{
        \item Il sistema è attivo e funzionante
        \item Il dataset corrente non è vuoto
    }

    \post{
        \item Il dataset corrente viene archiviato nel sistema
    }
    
    \actor{Utente}

    \subactors{}

    \trigger{L'utente deve archiviare il dataset corrente per renderlo persistente}
    
    \inc{\hyperref[subsec:UC-12]{UC-12}}

    \base{}

    \scenario{
        \item L'utente richiede l'archiviazione del dataset corrente.
        \item \texttt{<<include:UC-12>>}
    }

    \subscenario{
        \item[1.1] \textbf{Il dataset corrente contiene almeno una coppia con domanda e/o risposta vuota}
        \begin{itemize}
            \item[a.] \hyperref[subsec:UC-11]{UC-11}
        \end{itemize}
    }
\end{usecase}


\subsection{UC-10}
\label{subsec:UC-10}

\begin{usecase}{UC-10}{Sovrascrittura di un dataset archiviato}
    \req{\hyperref[item:RU-3]{RU-3}} 

    \pre{
        \item Il sistema è attivo e funzionante
        \item Il dataset corrente possiede già una versione archiviata
    }

    \post{
        \item La versione archiviata del dataset corrente viene sovrascritta
    }
    
    \actor{Utente}

    \subactors{}

    \trigger{L'utente deve aggiornare la versione archiviata di un dataset}
    
    \inc{\hyperref[subsec:UC-12]{UC-12}}

    \base{}

    \scenario{
        \item L'utente richiede l'archiviazione del dataset corrente che possiede già una versione archiviata.
        \item L'utente conferma la sovrascrittura
        \item \texttt{<<include:UC-12>>}
    }

    \subscenario{
        \item[2.1] \textbf{L'utente non conferma la sovrascrittura}
        \begin{itemize}
            \item [a.] L'operazione di sovrascrittura viene interrotta
        \end{itemize}
    }
\end{usecase}

\subsection{UC-11}
\label{subsec:UC-11}

\begin{usecase}{UC-11}{Visualizzazione errore di salvataggio}
    \req{} 

    \pre{
        \item Il sistema è attivo e funzionante
    }

    \post{
        \item L'utente conosce la causa del errore di salvataggio
    }
    
    \actor{Utente}

    \subactors{}

    \trigger{Il sistema riscontra un errore nel salvataggio del dataset o delle informazioni a esso associate}
    
    \inc{}

    \base{}

    \scenario{
        \item Viene notificato all'utente l'errore che ha impedito il corretto salvataggio
    }

\end{usecase}

\subsection{UC-12}
\label{subsec:UC-12}

\begin{figure}[H]
    \includegraphics{Sezioni/UseCase/Immagini/UC-12.pdf}
    \caption{Diagramma UC-12.}
\end{figure}

\begin{usecase}{UC-12}{Eliminazione di un dataset archiviato}

    \req{\hyperref[item:RU-3]{RU-3}} 

    \pre{
        \item Il sistema è attivo e funzionante
        \item Il dataset archiviato da eliminare esiste
    }

    \post{
        \item Il dataset selezionato viene eliminato 
    }
    
    \actor{Utente}

    \subactors{}

    \trigger{L'utente deve eliminare un dataset archiviato}
    
    \inc{}

    \base{}

    \scenario{
        \item L'utente richiede l'eliminazione di un dataset archiviato
        \item L'utente conferma l'eliminazione del dataset selezionato
        \item Il dataset viene eliminato dal sistema
    }

    \subscenario{
        \item[2.1] \textbf{L'utente annulla l'eliminazione del dataset selezionato}
        \begin{itemize}
            \item[a.] Il dataset selezionato non viene eliminato
            \item[b.] Viene interrotta l'operazione di eliminazione
        \end{itemize}
        \item[3.1] \textbf{L'eliminazione del dataset dal sistema produce un errore}
        \begin{itemize}
        \item[a.] Viene notificato all'utente l'errore che ha impedito la corretta eliminazione del dataset selezionato
        \end{itemize}
    }

\end{usecase}

\subsection{UC-13}
\label{subsec:UC-13}

\begin{usecase}{UC-13}{Copia di un dataset archiviato}
    \req{} 

    \pre{
        \item Il sistema è attivo e funzionante
        \item Il dataset da copiare esiste
    }

    \post{
        \item Viene archiviata una nuova copia del dataset selezionato usando un nome di default
    }
    
    \actor{Utente}

    \subactors{}

    \trigger{L'utente deve creare una copia di un dataset precedentemente archiviato}
    
    \inc{}

    \base{}

    \scenario{
        \item L'utente richiede di copiare un dataset archiviato
        \item Viene salvata nel sistema una copia del dataset selezionato con un nome di default
    }

    \subscenario{
        \item[2.1] \textbf{Avviene un errore durante il salvataggio del dataset caricato}
        \begin{itemize}
            \item[a.] \hyperref[subsec:UC-23]{UC-23}
        \end{itemize}
    }

\end{usecase}

\subsection{UC-14}
\label{subsec:UC-14}

\begin{figure}[H]
    \includegraphics{Sezioni/UseCase/Immagini/UC-14.pdf}
    \caption{Diagramma UC-14.}
\end{figure}

\begin{usecase}{UC-14}{Caricamento di un dataset archiviato}

    \req{\hyperref[item:RU-4]{RU-4}} 

    \pre{
        \item Il sistema è attivo e funzionante
        \item Il dataset da caricare è stato precedentemente archiviato
    }

    \post{
        \item Il dataset viene caricato
    }
    
    \actor{Utente}

    \subactors{}

    \trigger{L'utente deve caricare un dataset tra quelli archiviati nel sistema}
    
    \inc{}

    \base{}

    \scenario{
        \item L'utente richiede di caricare un dataset archiviato nel sistema
        \item L'utente seleziona il nome del dataset tra quelli archiviati
        \item Il dataset scelto viene caricato
    }

    \subscenario{
        \item[3.1] \textbf{Il dataset caricato non possiede una versione aggiornata archiviata e non è vuoto}
        \begin{itemize}
            \item[a.] \hyperref[subsec:UC-15]{UC-15}
        \end{itemize}
    }
\end{usecase}

\subsection{UC-15}
\label{subsec:UC-15}


\begin{figure}[H]
    \includegraphics[scale=0.60]{Sezioni/UseCase/Immagini/UC-15.pdf}
    \caption{Diagramma UC-15.}
\end{figure}

\begin{usecase}{UC-15}{Archiviazione del dataset corrente}
    
    \req{\hyperref[item:RU-5]{RU-5}} 

    \pre{
        \item Il sistema è attivo e funzionante
        \item Il dataset da archiviare è stato caricato come dataset corrente
        \item Il dataset corrente non è vuoto
        \item Il dataset corrente non possiede una copia archiviata aggiornata
    }

    \post{
        \item Il dataset corrente viene archiviato nel sistema
    }
    
    \actor{Utente}

    \subactors{}

    \trigger{L'utente deve archiviare il dataset corrente per renderlo persistente}
    
    \inc{\hyperref[subsec:UC-16]{UC-16}}

    \base{}

    \scenario{
        \item L'utente richiede l'archiviazione del dataset corrente.
        \item \texttt{<<include:UC-16>>}
    }

    \subscenario{
        \item[1.1] \textbf{Il dataset corrente contiene almeno una coppia con domanda e/o risposta vuota}
        \begin{itemize}
            \item[a.] \hyperref[subsec:UC-17]{UC-17}
        \end{itemize}
        \item[1.2] \textbf{Il dataset corrente è temporaneo perchè è stato appena creato}
        \begin{itemize}
            \item[a.] \hyperref[subsec:UC-18]{UC-18}
        \end{itemize}
    }
\end{usecase}


\subsection{UC-16}
\label{subsec:UC-16}

\begin{figure}[H]
    \includegraphics[scale=0.8]{Sezioni/UseCase/Immagini/UC-16.pdf}
    \caption{Diagramma UC-16.}
\end{figure}

\begin{usecase}{UC-16}{Esecuzione del test sul dataset corrente}

    \req{\hyperref[item:RU-5]{RU-5}} 

    \pre{
        \item Il sistema è attivo e funzionante
        \item Il dataset corrente non è vuoto
    }

    \post{
        \item L'utente conosce l'esito del test
    }
    
    \actor{Utente}

    \subactors{LLM}

    \trigger{L'utente deve testare il LLM usando il dataset corrente}
    
    \inc{\hyperref[subsec:UC-17]{UC-17}}

    \base{}

    \scenario{
        \item L'utente richiede l'esecuzione del test
        \item Per ogni coppia del dataset corrente si ottiene la risposta prodotta dal LLM sotto test
        \item Per ogni coppia del dataset corrente si calcola il grado di somiglianza tra la risposta prodotta dal LLM sotto test rispetto alla risposta attesa
        \item \texttt{<<include:UC-17>>}
    }

    \subscenario{
        \item[1.1] \textbf{Il dataset corrente contiene almeno una coppia invalida}
        \begin{itemize}
            \item[a.] hyperref[subsec:UC-11]{UC-11}
        \end{itemize}
        \item[2.1] \textbf{La comunicazione con LLM produce un errore}
        \begin{itemize}
        \item[a.] \hyperref[subsec:UC-18]{UC-18}
        \end{itemize}
    }
\end{usecase}

\subsection{UC-17}
\label{subsec:UC-17}

\begin{figure}[H]
    \includegraphics{Sezioni/UseCase/Immagini/UC-17.pdf}
    \caption{Diagramma UC-17.}
\end{figure}

\begin{usecase}{UC-17}{Visualizzazione dei risultati di un test}

    \req{\hyperref[item:RU-6]{RU-6}} 

    \pre{
        \item Il sistema è attivo e funzionante
        \item Sono disponibili i risultati di un test
    }

    \post{
        \item L'utente conosce l'esito del test
    }
    
    \actor{Utente}

    \subactors{}

    \trigger{L'utente deve testare il LLM usando il dataset corrente}
    
    \inc{\hyperref[subsec:UC-19]{UC-19}, \hyperref[subsec:UC-20]{UC-20}}

    \base{}

    \scenario{
        \item \texttt{<<include:UC-19>>}
        \item \texttt{<<include:UC-20>>}
    }

\end{usecase}

\subsection{UC-18}
\label{subsec:UC-18}

\begin{usecase}{UC-18}{Visualizzazione errore di comunicazione con LLM}

    \req{} 

    \pre{
        \item Il sistema è attivo e funzionante
        \item La comunicazione con il LLM non è andata a buon fine
    }

    \post{
        \item L'utente conosce l'errore di comunicazione
    }
    
    \actor{Utente}

    \subactors{LLM}

    \trigger{La comunicazione con il LLM produce un errore}
    
    \inc{}

    \base{}

    \scenario{
        \item Viene mostrato un messaggio di errore all'utente
    }

\end{usecase}

\subsection{UC-19}
\label{subsec:UC-19}

\begin{figure}[H]
    \includegraphics{Sezioni/UseCase/Immagini/UC-19.pdf}
    \caption{Diagramma UC-19.}
\end{figure}

\begin{usecase}{UC-19}{Caricamento come dataset corrente di un dataset archiviato}

    \req{\hyperref[item:RU-6]{RU-6}} 

    \pre{
        \item Il sistema è attivo e funzionante
        \item Il dataset da caricare come dataset corrente è stato precedentemente archiviato
    }

    \post{
        \item Il dataset viene caricato come dataset corrente
    }
    
    \actor{Utente}

    \subactors{}

    \trigger{L'utente deve caricare come dataset corrente uno tra quelli archiviati}
    
    \inc{}

    \base{}

    \scenario{
        \item L'utente richiede di caricare un dataset archiviato
        \item Il dataset scelto viene caricato come dataset corrente
    }

    \subscenario{
        \item[2.1] \textbf{Il dataset caricato non possiede una versione aggiornata archiviata e non è vuoto}
        \begin{itemize}
            \item[a.] \hyperref[subsec:UC-20]{UC-20}
        \end{itemize}
    }
\end{usecase}

\subsection{UC-20}
\label{subsec:UC-20}

\begin{figure}[H]
    \includegraphics[scale=0.8]{Sezioni/UseCase/Immagini/UC-20.pdf}
    \caption{Diagramma UC-20.}
\end{figure}

\begin{usecase}{UC-20}{Esecuzione del test sul dataset caricato}

    \req{\hyperref[item:RU-7]{RU-7}} 

    \pre{
        \item Il sistema è attivo e funzionante
        \item Il dataset caricato non è vuoto
    }

    \post{
        \item L'utente conosce l'esito del test
    }
    
    \actor{Utente}

    \subactors{LLM}

    \trigger{L'utente deve testare il LLM usando il dataset caricato}
    
    \inc{\hyperref[subsec:UC-22]{UC-22}}

    \base{}

    \scenario{
        \item L'utente richiede l'esecuzione del test
        \item Per ogni coppia del dataset caricato si ottiene la risposta prodotta dal LLM sotto test
        \item Per ogni coppia del dataset caricato si calcola il grado di somiglianza tra la risposta prodotta dal LLM sotto test rispetto alla risposta attesa
        \item \texttt{<<include:UC-22>>}
    }

    \subscenario{
        \item[1.1] \textbf{Il dataset caricato contiene almeno una coppia invalida}
        \begin{itemize}
            \item[a.] hyperref[subsec:UC-17]{UC-17}
        \end{itemize}
        \item[2.1] \textbf{La comunicazione con LLM produce un errore}
        \begin{itemize}
        \item[a.] \hyperref[subsec:UC-21]{UC-21}
        \end{itemize}
    }
\end{usecase}

\subsection{UC-21}
\label{subsec:UC-21}

\begin{figure}[H]
    \includegraphics{Sezioni/UseCase/Immagini/UC-21.pdf}
    \caption{Diagramma UC-21.}
\end{figure}

\begin{usecase}{UC-21}{Interazione con il grafico di dispersione}

    \req{\hyperref[item:RU-6]{RU-6}} 

    \pre{
        \item Il sistema è attivo e funzionante
        \item L'utente dispone del grafico di dispersione prodotto nella visualizzazione dei risultati del test
    }

    \post{
        \item Viene evidenziato il risultato relativo al punto del grafico di dispersione coinvolto nell'interazione.
    }
    
    \actor{Utente}

    \subactors{LLM}

    \trigger{L'utente ha la necessità di visualizzare un preciso risultato}
    
    \inc{}

    \base{}

    \scenario{
        \item L'utente interagisce con un punto del grafico di dispersione.
        
        
        \item Viene evidenziato l'elemento della lista dei risultati corrispondente al punto coinvolto nell'interazione.
    }

\end{usecase}

\subsection{UC-22}
\label{subsec:UC-22}

\begin{usecase}{UC-22}{Eliminazione di un dataset archiviato}

    \req{\hyperref[item:RU-6]{RU-6}} 

    \pre{
        \item Il sistema è attivo e funzionante
        \item Il dataset archiviato da eliminare esiste
    }

    \post{
        \item Il dataset selezionato viene eliminato 
    }
    
    \actor{Utente}

    \subactors{}

    \trigger{L'utente deve eliminare un dataset archiviato}
    
    \inc{}

    \base{}

    \scenario{
        \item L'utente richiede l'eliminazione di un dataset archiviato
        \item L'utente conferma l'eliminazione del dataset selezionato
        \item Il dataset viene eliminato dal sistema
    }

    \subscenario{
        \item[2.1] \textbf{L'utente annulla l'eliminazione del dataset selezionato}
        \begin{itemize}
            \item[a.] Il dataset selezionato non viene eliminato
            \item[b.] Viene interrotta l'operazione di eliminazione
        \end{itemize}
        \item[3.1] \textbf{L'eliminazione del dataset dal sistema produce un errore}
        \begin{itemize}
        \item[a.] Viene notificato all'utente l'errore che ha impedito la corretta eliminazione del dataset selezionato
        \end{itemize}
    }

\end{usecase}

\subsection{UC-23}
\label{subsec:UC-23}

\begin{figure}[H]
    \includegraphics{Sezioni/UseCase/Immagini/UC-23.pdf}
    \caption{Diagramma UC-23.}
\end{figure}

\begin{usecase}{UC-23}{Visualizzazione dei risultati di un test}

    \req{\hyperref[item:RU-8]{RU-8}} 

    \pre{
        \item Il sistema è attivo e funzionante
        \item Sono disponibili i risultati di un test
    }

    \post{
        \item L'utente conosce l'esito del test
    }
    
    \actor{Utente}

    \subactors{}

    \trigger{L'utente deve testare il LLM usando il dataset caricato}
    
    \inc{\hyperref[subsec:UC-24]{UC-24}, \hyperref[subsec:UC-25]{UC-25}}

    \base{}

    \scenario{
        \item \texttt{<<include:UC-24>>}
        \item \texttt{<<include:UC-25>>}
    }

\end{usecase}

