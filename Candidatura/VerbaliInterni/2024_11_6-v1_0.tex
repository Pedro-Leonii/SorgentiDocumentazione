\documentclass[a4paper, 12pt]{article}

\usepackage{custom}


%--------------------VARIABILI--------------------
\def\lastversion{v1.0}
\def\title{Verbale interno}
\def\date{6 novembre 2024}
%-------------------------------------------------

\begin{document}

\primapagina

\begin{registromodifiche}
        v1.0 & 11 novembre 2024& Pedro Leoni& & Accettazione documento\\
    \hline 
        v0.1 & 7 novembre 2024& Peric Marko& Pedro Leoni & Prima Stesura\\
    \hline 
\end{registromodifiche}


\tableofcontents

\newpage

\section{Registro presenze}
\begin{itemize}
    \item[] \textbf{Data}: 6 novembre 2024
    \item[] \textbf{Ora inizio}:  14:30
    \item[] \textbf{Ora fine}: 15:30
    \item[] \textbf{Piattaforma}: Discord	
\end{itemize}
\begin{table}[!h]
\centering
{\renewcommand{\arraystretch}{2}
\begin{tabularx}{\textwidth}{| X | X |}
    \hline
        \textbf{\large Componente} & 
        \textbf{\large Presenza} \\ 
    \hline 
    \hline
        Eghosa Matteo Igbinedion Osamwonyi& Presente \\
    \hline 
        Guirong Lan& Presente \\
    \hline 
        Enrico Bianchi& Presente \\
    \hline 
        Francesco Savio& Presente \\
    \hline 
        Marko Peric& Presente \\
    \hline 
        Pedro Leoni& Presente \\
    \hline 

\end{tabularx}}
\end{table}

\newpage

\section{Verbale}
\subsection{Argomenti trattati}
Il gruppo si è riunito per decidere su come procedere con la scelta del capitolato. Sono state discusse le seguenti domande:
\begin{itemize}
    \item Ricandidarsi al capitolato C7 o sceglierne un altro.
    \item Quale capitolato scegliere e perché.
    \item Ristrutturazione del documento \textit{Norme di Progetto} aggiungendo in particolare informazioni più specifiche su issue e ciclo di vita dei documenti.
    \item Ristrutturazione del Way of Working.
\end{itemize}
\subsection{Decisioni prese}
\begin{itemize}
    \item Il gruppo ha deciso di provare a candidarsi al capitolato C1 per ottimizzare i tempi di sviluppo.
    \item Il gruppo ha deciso di modificare le Norme di Progetto cercando di sfruttare lo standard \textbf{ISO-12207}.
    \item Il gruppo ha organizzato la riunione con l'azienda Zucchetti.
\end{itemize}
\section{To Do}
\begin{itemize}
\item Terminare la issue \textbf{\#60}, stesura del verbale interno del 6 novembre 2024.
\item Terminare l'issue \textbf{\#66} composta dalle sotto issue \textbf{\#52},\textbf{\#53},\textbf{\#54},\textbf{\#59},\textbf{\#61},\textbf{\#61} che riguardano
ristrutturazione del documento \textit{Norme di Progetto}.
\item Terminare la issue \textbf{\#64}, correzzione della lettera di candidatura.
\end{itemize}

\end{document}