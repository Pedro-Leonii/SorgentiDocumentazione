\documentclass[a4paper, 12pt]{article}

\usepackage{custom}


%--------------------VARIABILI--------------------
\def\lastversion{v0.1}
\def\title{Verbale interno}
\def\data-verbale{13 dicembre 2024}
%------------------------------------------------

\begin{document}

\primapagina

\begin{registromodifiche}
        v0.1 & 13 dicembre 2024 & Eghosa Matteo Igbinedion Osamwonyi&& Prima stesura del documento\\
    \hline 
\end{registromodifiche}

\tableofcontents

\newpage

\section{Registro presenze}
\begin{itemize}
    \item[] \textbf{Data}: \data-verbale
    \item[] \textbf{Ora inizio}:  15:30
    \item[] \textbf{Ora fine}: 16:30
    \item[] \textbf{Piattaforma}: Discord	
\end{itemize}
\begin{table}[!h]
\centering
{\renewcommand{\arraystretch}{2}
\begin{tabularx}{\textwidth}{| X | X |}
    \hline
        \textbf{\large Componente} & 
        \textbf{\large Presenza} \\
    \hline 
    \hline
        Eghosa Matteo Igbinedion Osamwonyi&
        Presente \\
    \hline 
        Guirong Lan&
        Presente \\
    \hline 
        Enrico Bianchi&
        Presente \\
    \hline 
        Francesco Savio&
        Presente \\
    \hline 
        Marko Peric&
        Presente \\
    \hline 
        Pedro Leoni&
        Presente \\
    \hline 

\end{tabularx}}
\end{table}

\newpage

\section{Verbale}
\subsection{Argomenti trattati}
\begin{itemize}
    \item Definizione degli obiettivi del prodotto, delle funzioni del prodotto e delle caratteristiche utente
    \item Stesura dei processi primari da includere nelle norme di progetto
    \item Avvio del primo periodo:
    \begin{itemize}
        \item Redazione del file \texttt{sprint1.tex}
        \item Creazione del file \texttt{primoperiodo.tex}, contenente ore disponibili prima dell'ordine e ore rimaste per ruolo
    \end{itemize}
    \item Stesura del piano di progetto con le seguenti sezioni:
    \begin{itemize}
        \item Analisi dei rischi
        \item Stima dei costi
        \item Milestone principali
    \end{itemize}
    \item Creazione di una GitHub Action per il calcolo automatico delle ore di lavoro per ogni issue
\end{itemize}

\subsection{Decisioni prese}
\begin{itemize}
    \item Procedere con la definizione dettagliata degli obiettivi, funzioni e caratteristiche utente del prodotto
    \item Inserire i processi primari nelle norme di progetto
    \item Iniziare il primo periodo con la scrittura di \texttt{sprint1.tex} e \texttt{primoperiodo.tex}
    \item Completare la stesura del piano di progetto includendo analisi dei rischi, stima dei costi e milestone principali
    \item Sviluppare una GitHub Action per il calcolo delle ore di lavoro per issue
\end{itemize}

\section{To Do}
\begin{itemize}
    \item Redigere la definizione di obiettivi, funzioni e caratteristiche utente del prodotto
    \item Scrivere i processi primari da includere nelle norme di progetto
    \item Creare e completare i file \texttt{sprint1.tex} e \texttt{primoperiodo.tex}
    \item Stendere il piano di progetto con analisi dei rischi, stima dei costi e milestone principali
    \item Implementare la GitHub Action per il calcolo automatico delle ore di lavoro
\end{itemize}

\end{document}
