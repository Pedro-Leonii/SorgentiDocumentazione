\documentclass[a4paper, 12pt]{article}

\usepackage{custom}

%--------------------VARIABILI--------------------
\def\lastversion{v1.0}
\def\title{Verbale interno}
\def\date{4 novembre 2024}
%-------------------------------------------------

\begin{document}

\primapagina


\begin{registromodifiche}
        v1.0 & 5 novembre 2024 & Eghosa Igbinedion & & Approvazione documento \\
    \hline 
        v0.1 & 4 novembre 2024 & Marko Peric & Guirong Lan & Prima stesura\\
    \hline 
\end{registromodifiche}

\tableofcontents

\newpage

\section{Registro presenze}
\begin{itemize}
    \item[] \textbf{Data}: 4 novembre 2024
    \item[] \textbf{Ora inizio}:  14:30
    \item[] \textbf{Ora fine}: 16:30
    \item[] \textbf{Piattaforma}: Discord	
\end{itemize}
\begin{table}[!h]
\centering
{\renewcommand{\arraystretch}{2}
\begin{tabularx}{\textwidth}{| X | X |}
    \hline
        \textbf{\large Componente} & 
        \textbf{\large Presenza} \\ 
    \hline 
    \hline
        Eghosa Matteo Igbinedion Osamwonyi& Presente \\
    \hline 
        Guirong Lan& Presente \\
    \hline 
        Enrico Bianchi& Presente \\
    \hline 
        Francesco Savio& Presente \\
    \hline 
        Marko Peric& Presente \\
    \hline 
        Pedro Leoni& Presente \\
    \hline 

\end{tabularx}}
\end{table}

\newpage

\section{Verbale}
\subsection{Argomenti trattati}
Il gruppo si è riunito per rispondere alle seguenti domande:
\begin{itemize}
    \item A quale capitolato candidarci e perché?
    \item Come migliorare sui punti indicati come carenti dal professore?
    \item Come suddividerci il lavoro?
    \item Cosa bisogna fare in maniera urgente e cosa no?
\end{itemize}
\subsection{Decisioni prese}
Discutendo riguardo ai temi soprastanti il gruppo è giunto alle seguenti decisioni:
\begin{itemize} 
    \item Il gruppo ha optato per il capitolato C5, motivo del quale indicato nella lettera di presentazione.
    \item Per colmare le lacune segnalate dal docente: 
    \begin{itemize} 
        \item Ogni modifica sarà ora soggetta a verifica da parte di un membro apposito, senza che questi intervenga direttamente.        
        \item La redazione dei verbali sarà più sintetica, favorendo chiarezza e concisione. 
    \end{itemize} 
    \item I ruoli sono stati suddivisi in modo equo, come indicato nel documento di preventivo dei costi. 
\end{itemize}

\section{To Do}
\begin{itemize}
    \item Terminare la issue \textbf{\#42}: stesura del verbale interno del 2024/11/4.
    \item Terminare la issue \textbf{\#39}: modifica dei template dei documenti.
    \item Terminare la issue \textbf{\#41}: modifica della lettera di presentazione.
    \item Terminare la issue \textbf{\#40}: modifica del documento di preventivo dei costi.
\end{itemize}
\end{document}