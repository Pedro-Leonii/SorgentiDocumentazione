\documentclass[a4paper, 12pt]{article}

\usepackage[italian]{babel}
\usepackage{tikz}
\usepackage{xcolor}
\usepackage{graphicx}
\usepackage{hyperref}
\usepackage{imakeidx}
\usepackage{caption}
\usepackage{fancyhdr}
\usepackage{geometry}
\usepackage{tabularx}


%--------------------VARIABILI--------------------
\def\logo{../../Immagini/logo.jpeg}
\def\ultima-versione{0.1}
\def\titolo{Verbale interno }
\def\data-verbale{2024/11/6}
%------------------------------------------------

\usetikzlibrary{calc}
\definecolor{fp-blue}{HTML}{2885c8}
\definecolor{fp-red}{HTML}{ea5f64}
\makeindex[title=Indice]
\hypersetup{hidelinks}

\pagestyle{fancy}
\fancyhead[L]{}
\setlength{\headheight}{15pt}
\fancyhead[R]{\titolo - \ultima-versione}

\renewcommand{\familydefault}{\sfdefault}
\newcommand{\glossario}[1]{\fontfamily{lmr}\selectfont{\textit{#1\textsubscript{\small G}}}}
\newcolumntype{C}{>{\centering\arraybackslash}X}

%--------------------INFORMAZIONI PRIMA PAGINA-------------------- 
\title{\Huge \textbf{\titolo}}
\author{\Large{Alt} \raisebox{0.3ex}{\normalsize  +} \Large{F4}}
\date{\data-verbale}
%----------------------------------------------------------------

\begin{document}

\begin{titlepage}      
    \maketitle
    \thispagestyle{empty}  

    \begin{tikzpicture}[remember picture, overlay]
        \fill[fp-blue] 
        ($(current page.south west) + (0, 10)$) 
        -- ($(current page.center) + (0, -8)$)
        -- ($(current page.center) + (0, -15)$)
        -- (current page.south west);

        \fill[fp-red]
        ($(current page.south east) + (0, 10)$) 
        -- ($(current page.center) + (0, -8)$)
        -- ($(current page.center) + (0, -15)$)
        -- (current page.south east);

        \clip ($(current page.center) + (0, -8)$) circle (1cm) node 
        {\includegraphics[width=.25\textwidth]{\logo}};
        
    \end{tikzpicture}    
\end{titlepage}

\thispagestyle{plain}
\newgeometry{ignoreall, hmargin=20pt}
\begin{table}[!h]
    \centering
    \caption*{\textbf{\Large Registro Modifiche}}
    {\renewcommand{\arraystretch}{2}
    \begin{tabularx}{\textwidth}{| c | c | C | C | C |}
        \hline
            \textbf{\normalsize Versione} & 
            \textbf{\normalsize Data} & 
            \textbf{\normalsize Autore/i} & 
            \textbf{\normalsize Verificatore} &
            \textbf{\normalsize Descrizione} \\ 
        \hline \hline
        \ultima-versione & 
        2024/11/7 & 
        Peric Marko&
        & 
        Prima Stesura\\
        \hline 
    \end{tabularx}}
\end{table}
\restoregeometry

\tableofcontents

\newpage

\section{Registro presenze}
\begin{itemize}
    \item[] \textbf{Data}: \data-verbale
    \item[] \textbf{Ora inizio}:  14:30
    \item[] \textbf{Ora fine}: 15:30
    \item[] \textbf{Piattaforma}: Discord	
\end{itemize}
\begin{table}[!h]
\centering
{\renewcommand{\arraystretch}{2}
\begin{tabularx}{\textwidth}{| X | X |}
    \hline
        \textbf{\large Componente} & 
        \textbf{\large Presenza} \\ 
    \hline 
    \hline
        Eghosa Matteo Igbinedion Osamwonyi&
        Presente \\
    \hline 
        Guirong Lan&
        Presente \\
    \hline 
        Enrico Bianchi&
        Presente \\
    \hline 
        Francesco Savio&
        Presente \\
    \hline 
        Marko Peric&
        Presente \\
    \hline 
        Pedro Leoni&
        Presente \\
    \hline 

\end{tabularx}}
\end{table}

\newpage

\section{Verbale}
\subsection{Argomenti trattati}
Il gruppo si è riunito per decidere su come procedere con la scelta del capitolato. Sono state discusse le seguenti domande:
\begin{itemize}
    \item Ricandidarsi al capitolato C7 o sceglierne un altro.
    \item Quale capitolato scegliere e perché.
    \item Ristrutturazione del documento \textit{Norme di Progetto}.
    \item Ristrutturazione del Way of Working.
\end{itemize}
\subsection{Decisioni prese}
\begin{itemize}
    \item Il gruppo ha deciso di provare a candidarsi al capitolato C1 per ottimizzare i tempi di sviluppo.
    \item Il gruppo ha deciso di modificare le Norme di Progetto cercando di sfruttare lo standard \textbf{ISO-12207}.
    \item Il gruppo ha organizzato la riunione con l'azienda Zucchetti.
\end{itemize}
\section{To Do}
\begin{itemize}
\item Terminare la issue \textbf{\#60}, stesura del verbale interno del 2024/11/6.
\item Terminare le issue \textbf{\#52},\textbf{\#53},\textbf{\#54},\textbf{\#59},\textbf{\#61},\textbf{\#61} di
ristrutturazione del documento \textit{Norme di Progetto}.
\item Terminare la issue \textbf{\#64}, correzzione della lettera di candidatura.
\end{itemize}

\end{document}