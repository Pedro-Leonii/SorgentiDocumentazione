\documentclass[a4paper, 12pt]{article}

\usepackage[italian]{babel}
\usepackage{tikz}
\usepackage{xcolor}
\usepackage{graphicx}
\usepackage{hyperref}
\usepackage{imakeidx}
\usepackage{caption}
\usepackage{fancyhdr}
\usepackage{tabularx}
\usepackage{geometry}


%--------------------VARIABILI--------------------
\def\logo{../Immagini/logo.jpeg}
\def\ultima-versione{v3.0}
\def\titolo{Valutazione Capitolati}
%------------------------------------------------

\usetikzlibrary{calc}
\definecolor{fp-blue}{HTML}{2885c8}
\definecolor{fp-red}{HTML}{ea5f64}
\makeindex[title=Indice]
\hypersetup{
    hidelinks,
    colorlinks=true,
    linkcolor=fp-red,
    filecolor=magenta,      
    urlcolor=fp-blue,
    pdfpagemode=FullScreen,
}

\pagestyle{fancy}
\fancyhead[L]{}
\setlength{\headheight}{15pt}
\fancyhead[R]{\titolo \space - \ultima-versione}

\renewcommand{\familydefault}{\sfdefault}
\newcommand{\glossario}[1]{\fontfamily{lmr}\selectfont{\textit{#1\textsubscript{\small G}}}}
\newcolumntype{C}{>{\centering\arraybackslash}X}

%--------------------INFORMAZIONI PRIMA PAGINA-------------------- 
\title{\Huge \textbf{\titolo}}
\author{\Large{Alt} \raisebox{0.3ex}{\normalsize  +} \Large{F4}}
\date{11 novembre 2024}
%----------------------------------------------------------------

\begin{document}

\begin{titlepage}      
    \maketitle
    \thispagestyle{empty}  

    \begin{tikzpicture}[remember picture, overlay]
        \fill[fp-blue] 
        ($(current page.south west) + (0, 10)$) 
        -- ($(current page.center) + (0, -8)$)
        -- ($(current page.center) + (0, -15)$)
        -- (current page.south west);

        \fill[fp-red]
        ($(current page.south east) + (0, 10)$) 
        -- ($(current page.center) + (0, -8)$)
        -- ($(current page.center) + (0, -15)$)
        -- (current page.south east);

        \clip ($(current page.center) + (0, -8)$) circle (1cm) node 
        {\includegraphics[width=.25\textwidth]{\logo}};
        
    \end{tikzpicture}    
\end{titlepage}

\thispagestyle{plain}
\newgeometry{ignoreall, hmargin=20pt}
\begin{table}[!h]
    \centering
    \caption*{\textbf{\Large Registro Modifiche}}
    {\renewcommand{\arraystretch}{2}
    \begin{tabularx}{\textwidth}{| c | c | C | C | C |}
        \hline
            \textbf{\normalsize Versione} & 
            \textbf{\normalsize Data} & 
            \textbf{\normalsize Autore/i} & 
            \textbf{\normalsize Verificatore} &
            \textbf{\normalsize Descrizione} \\ 
        \hline \hline
        \ultima-versione & 
            11 novembre 2024 & 
            Pedro Leoni &
            & 
            Approvazione documento \\ 
        \hline 
        v2.1 & 
        11 novembre 2024 & 
        Pedro Leoni &
        Enrico Bianchi& 
        Modifica sezione \hyperref[sec:caps]{Capitolato scelto} \\ 
        \hline
        v2.0 & 
        5 novembre 2024 & 
        Francesco Savio &
        & 
        Approvazione documento \\ 
    \hline 
        v1.1 & 
        5 novembre 2024 & 
        Marko Peric & 
        Pedro Leoni&
        Modifica sezione \hyperref[sec:caps]{Capitolato scelto} \\ 
        \hline
            v1.0 & 
            31 ottobre 2024 & 
            Guirong Lan &
            & 
            Approvazione documento \\ 
        \hline 
            v0.1 & 
            31 ottobre 2024 & 
            \raggedright Pedro Leoni, Francesco Savio, Marko Peric e Eghosa Matteo Igbinedion Osamwonyi & 
            Enrico Bianchi&
            Prima stesura \\ 
        \hline 
    \end{tabularx}}
\end{table}
\restoregeometry

\tableofcontents

\newpage

\section{Introduzione}
Il presente documento ha lo scopo di fornire una valutazione dei capitolati esposti dalle aziende proponenti.
Per ciascun capitolato, verrà fornita una descrizione sintetica, accompagnata da un’analisi dei punti di forza e delle eventuali criticità riscontrate.

\section{Capitolato scelto}
\label{sec:caps}

\subsection{C1 / Artificial QI}
\subsubsection{Descrizione}
\begin{itemize}
    \item \textbf{Proponente}: Zucchetti
    \item \textbf{Committenti}: Tullio Vardanega e Riccardo Cardin
    \item \textbf{Obiettivo}: sviluppare un sistema per valutare l'efficacia dei sistemi basati su Large Language Models(LLM) nel rispondere a domande indicate dall’utente. 
    Questo strumento permetterà agli sviluppatori di analizzare le conseguenze delle loro scelte riguardanti i modelli utilizzati diminuendo le verifiche umane.     
\end{itemize}

\subsubsection{Dominio Applicativo}
Il progetto ha come obiettivo quello di fornire un metodo per valutare la capacità di risposta dei Large Language Models(LLM) e pertanto può essere applicato a ogni azienda software che sviluppa sistemi che sfruttano questi modelli.

\subsubsection{Dominio Tecnologico}
La proponente non ha consigliato delle specifiche tecnologie tuttavia ha dato degli spunti sui seguenti temi:
\begin{itemize}
    \item Possibili metodi di valutazione delle risposte dei Large Language Models(LLM).
    \item Persistenza delle domande e delle risposte.
    \item Libreria per la ricerca semantica con esempi di \textit{fine tuning} di modelli o di creazione di modelli da zero.
    \item Large Language Model(LLM) che possono essere utilizzati.
\end{itemize}


\subsubsection{Aspetti Positivi}
\begin{itemize}
    \item Il capitolato pone ai gruppi un problema aperto molto interessante ovvero le modalità di valutazione dei LLM.
    \item L'obiettivo del progetto è attuale e stimolante.
\end{itemize}

\subsubsection{Aspetti Critici}
\begin{itemize}
    \item Il capitolato non fornisce molti consigli sulle tecnologie da utilizzare.
    \item La presentazione del capitolato è risultata sbrigativa al gruppo.
\end{itemize}

\subsection{Motivazione scelta}
\begin{itemize}
    \item Ultimo capitolato rimasto.
    \item L'argomento è attuale e stimolante.
\end{itemize}

\section{Altri capitolati}
\subsection{C2 / Vimar GENIALE}

\subsubsection{Descrizione}
\begin{itemize}
    \item \textbf{Proponente}: VIMAR
    \item \textbf{Committenti}: Tullio Vardanega e Riccardo Cardin
    \item \textbf{Obiettivo}: sviluppare un applicativo interattivo per installatori, in grado di fornire informazioni dettagliate e grafiche sui prodotti Vimar. 
    Questo strumento utilizzerà linguaggio naturale per consentire agli installatori di accedere rapidamente a dati tecnici, schemi elettrici e istruzioni per diversi tipi di impianti (tradizionali, smart e domotici).    
\end{itemize}

\subsubsection{Dominio Applicativo}
Il progetto mira a produrre un sistema in grado di fornire informazioni tecniche su diverse tipologie d'impianti elettrici e domotici di modo da semplificare l’installazione degli stessi.

\subsubsection{Dominio Tecnologico}
La proponente ha suggerito principalmente l’utilizzo delle seguenti tecnologie:
\begin{itemize}
    \item Docker e docker-compose per l’implementazione del principio \textit{Infrastructure as a code}.
    \item Approccio\textit{ Retrieval Argument Generation RAG}
    \item Python per lo sviluppo del back-end.
    \item PostgreSQL o alternative NoSQL come TimescaleDB per la persistenza dei dati ed in particolare l’utilizzo dell’estensione pgvector(se si usa PostreSQL) per realizzare gli embeddings da sfruttare con il componente d'interrogazione.
    \item Modelli LLM open source per l’implementazione del componente d'interrogazione come: LAMA, Mistral, Bert ecc.
    \item Angular, VueJS o Flask per lo sviluppo del front-end.
    \item Servizio AWS lightsail per eseguire le istanze dei container.
\end{itemize}

\subsubsection{Aspetti Positivi}
\begin{itemize}
    \item Il progetto utilizza tecnologie molto diffuse e innovative che possono essere utili nel mondo del lavoro.
    \item Capitolato presentato in modo convincente.
    \item La disponibilità espressa dal committente è alta.
    \item La proponente mette a disposizione un ambiente AWS ai gruppi.
\end{itemize}

\subsubsection{Aspetti Critici}
\begin{itemize}
    \item Il progetto pur essendo ritenuto dal gruppo molto interessante è apparso molto complesso.
    \item I posti disponibili sono stati fin da subito pochi rispetto al interesse generale.
\end{itemize}

\subsection{C3 / Automatizzare le routine digitali tramite l’intelligenza generativa}

\subsubsection{Descrizione}
\begin{itemize}
    \item \textbf{Proponente}: Var Group S.p.A
    \item \textbf{Committenti}: Tullio Vardanega e Riccardo Cardin
    \item \textbf{Obiettivo}: sviluppare un sistema che consenta di creare graficamente dei workflow descrivendo la logica delle attività da eseguire per ogni passo in linguaggio naturale.
    Le descrizioni fornite vengono usate per produrre le operazioni da eseguire mediante una AI generativa.    
\end{itemize}

\subsubsection{Dominio Applicativo}
Il sistema mira a fornire in automazione routine aziendali e un ottimizzazione di flussi di lavoro.

\subsubsection{Dominio Tecnologico}
La proponente ha suggerito principalmente l'utilizzo delle seguenti tecnologie:
\begin{itemize}
    \item La Proponente ha suggerito principalmente l’utilizzo delle seguenti tecnologie:
    \item Sistemi di AI generativa forniti da AWS.
    \item MongoDB database non relazionale per la persistenza dei dati 
    \item NodeJS o Python per il back-end.
    \item React per l’interfaccia grafica in ambiente Windows.
    \item Swift UI per l’interfaccia grafica in ambiente Apple.
\end{itemize}

\subsubsection{Aspetti Positivi}
\begin{itemize}
    \item La proponente è disposta a guidare l’apprendimento delle metodologie \textit{agili}, apprezzate nel mondo del lavoro.
    \item Le tecnologie proposte sono moderne e apprezzate nel mondo del lavoro.
\end{itemize}

\subsubsection{Aspetti Critici}
\begin{itemize}
    \item Difficoltà nel comprendere la parte di AI generativa a partire dal capitolato e la sua presentazione.
    \item Pochi posti disponibili e tanta richiesta.
\end{itemize}

\subsection{C4 / NearYou - Smart custom advertising platform}
\subsubsection{Descrizione}
\begin{itemize}
    \item \textbf{Proponente}: SyncLab 
    \item \textbf{Committenti}: Tullio Vardanega e Riccardo Cardin
    \item \textbf{Obiettivo}: Sviluppare una piattaforma per pubblicità personalizzata basata su IA, che genera annunci mirati per utenti, considerando la loro posizione geografica e altri dati personali. La piattaforma sfrutta LLM per creare messaggi pubblicitari contestuali e pertinenti.
\end{itemize}

\subsubsection{Dominio Applicativo}
Pubblicità digitale personalizzata in tempo reale basata sulle preferenze, la posizione geografica e le precedenti interazioni degli utenti che permette un maggior ritorno sull’investimento da parte dei brand data la maggior pertinenza delle pubblicità mostrate.

\subsubsection{Dominio Tecnologico}
La proponente ha suggerito principalmente l’utilizzo delle seguenti tecnologie:
\begin{itemize}
    \item Tecnologie per data streaming (Apache Kafka, RabbitMQ).
    \item Python per la simulazione dei dati.
    \item Utilizzo di un LLM mediante framework come LangChain.
    \item Strumento per lo stream processing(Apache Airflow, Apache NiFi o framework come Apache Spark e Apache Flink).
    \item Database in grado di gestire dati geospaziali(PosGIS, ClickHouse, Timescale).
    \item Strumenti di data visualization (Grafana, Superset).
\end{itemize}

\subsubsection{Aspetti Positivi}
\begin{itemize}
    \item Ampia libertà nella scelta delle tecnologie, promuovendo il ragionamento critico.
    \item La proponente fornisce supporto continuo e dettagli sulle tecnologie, favorendo lo sviluppo di competenze pratiche.
\end{itemize}

\subsubsection{Aspetti Critici}
\begin{itemize}
    \item L’opportunità che il progetto vuole colmare non è stata ritenuta così innovativa da parte del gruppo.
    \item Il progetto si focalizza principalmente sulla configurazione di strumenti già esistenti dando poco spazio alla creazione di componenti software da zero.
\end{itemize}

\subsection{C5 / 3Dataviz}
\subsubsection{Descrizione}
\begin{itemize}
    \item \textbf{Proponente}: Sanmarco Informatica
    \item \textbf{Committenti}: Tullio Vardanega e Riccardo Cardin
    \item \textbf{Obiettivo}: creare un'interfaccia web interattiva per la visualizzazione tridimensionale di dati in istogrammi. L’obiettivo è tradurre i dati in rappresentazioni visive che supportino decisioni strategiche, con una struttura a barre 3D, e funzionalità di rotazione, zoom e filtraggio dinamico.

\end{itemize}

\subsubsection{Dominio Applicativo}
Visualizzazione di dati complessi in contesti aziendali, come dati di vendita, meteo e statistiche di consumo, con possibilità di analisi approfondite.

\subsubsection{Dominio Tecnologico}
La proponente ha suggerito l’utilizzo delle seguenti tecnologie:
\begin{itemize}
    \item Three.js o  D3.js  per la visualizzazione e l'interazione coi grafici a barre.
    \item React o Angular per l’interfaccia grafica.
    \item Database o API terze per l’ottenimento dei dati da visualizzare.
\end{itemize}

\subsubsection{Aspetti Positivi}
\begin{itemize}
    \item Introduce l'uso di librerie avanzate per la grafica 3D.
    \item Tecnologie e strumenti attuali, utili per il mercato del lavoro.
\end{itemize}

\subsubsection{Aspetti Critici}
\begin{itemize}
    \item La programmazione 3D può sembrare particolarmente complessa(dopo l'incontro con la proponente il gruppo ha rivalutato questo punto).
\end{itemize}

\subsection{C6 / Sistema di gestione di un magazzino distribuito}
\subsubsection{Descrizione}
\begin{itemize}
    \item \textbf{Proponente}: M31 S.r.l
    \item \textbf{Committenti}: Tullio Vardanega e Riccardo Cardin
    \item \textbf{Obiettivo}: sviluppare un sistema distribuito per la gestione dell'inventario su una rete di magazzini. Il sistema deve garantire la sincronizzazione dei dati in tempo reale, il riassortimento predittivo tramite machine learning, la gestione centralizzata e la sicurezza dei dati.
\end{itemize}

\subsubsection{Dominio Applicativo}
Ottimizzazione della gestione degli inventari distribuiti in contesti logistici con nodi geograficamente separati.

\subsubsection{Dominio Tecnologico}
La proponente ha suggerito l’utilizzo delle seguenti tecnologie:
\begin{itemize}
    \item Nest.js per lo sviluppo dei microservizi.
    \item Go per lo sviluppo di componenti ad alte prestazioni come i servizi di sincronizzazione.
    \item Kafka o NATS per la comunicazione asincrona tra i microservizi.
    \item Google Cloud Platform per ospitare il sistema di orchestrazione basato su Kubernetes.
    \item MongoDB per la persistenza dei dati non strutturati.
    \item PostgreSQL per la gestione dei dati strutturati.
    \item Angular per l’interfaccia grafica
\end{itemize}

\subsubsection{Aspetti Positivi}
\begin{itemize}
    \item Affronta problematiche attuali come scalabilità e sincronizzazione dati.
    \item Uso di tecnologie e architetture moderne, come microservizi e machine learning, applicabili al mondo reale.
    \item La proponente offre supporto con un dataset e un esperto di logistica.
    \item Molto stimolante.
\end{itemize}

\subsubsection{Aspetti Critici}
\begin{itemize}
    \item Richiede esperienza in tecnologie distribuite e machine learning.
    \item Il progetto è stato ritenuto troppo complesso dal gruppo.
    \item Pochi posti a disposizione.
\end{itemize}

\subsection{C7 / LLM: Assistente virtuale}

\subsubsection{Descrizione}
\begin{itemize}
    \item \textbf{Proponente}: Ergon Informatica Srl
    \item \textbf{Committenti}: Tullio Vardanega e Riccardo Cardin
    \item \textbf{Obiettivo}: sviluppare un assistente virtuale per supportare i clienti nella ricerca d'informazioni sui prodotti, usando LLM per rispondere a domande frequenti.
\end{itemize}

\subsubsection{Dominio Applicativo}
Estrapolare le informazioni già presenti negli archivi digitali delle aziende e renderle fruibili direttamente all’utente finale che esegue richieste tramite il linguaggio naturale.

\subsubsection{Dominio Tecnologico}
La proponente ha suggerito l’utilizzo delle seguenti tecnologie:
\begin{itemize}
    \item Database relazionali (es. MySQL).
    \item Large Language Model(LLM) open-source (BLOOM, Falcon AI, Italia by iGenius).
    \item Interfaccia utente usando framework NET MAUI o Android.
\end{itemize}

\subsubsection{Aspetti Positivi}
\begin{itemize}
    \item Capitolato e presentazione convincenti.
    \item Utilizzo interessante dei Large Language Model(LLM).
    \item Il proponente fornisce i dati di un caso di studio da usare per lo sviluppo.
    \item Il proponente offre due corsi su piattaforme online che trattano i Large Language Model(LLM) e lo sviluppo tramite il framework .NET MAUI.
    \item Il proponente ha dichiarato ampio supporto.
\end{itemize}

\subsubsection{Aspetti Critici}
\begin{itemize}
    \item .NET MAUI non è una tecnologia che interessa molto al gruppo.
\end{itemize}

\subsection{C8 / Requirement Tracker - Plug-in VS Code}
\subsubsection{Descrizione}
\begin{itemize}
    \item \textbf{Proponente}: Bluewind Srl
    \item \textbf{Committenti}: Tullio Vardanega e Riccardo Cardin
    \item \textbf{Obiettivo}: creare un plug-in per Visual Studio Code che permetta il tracciamento automatico dei requisiti di progetto nel codice sorgente e migliori la qualità dei requisiti stessi.
\end{itemize}

\subsubsection{Dominio Applicativo}
Applicabile per il tracciamento e gestione dei requisiti nei progetti software embedded, integrando analisi del codice e suggerimenti per la scrittura dei requisiti.

\subsubsection{Dominio Tecnologico}
La proponente ha suggerito l’utilizzo delle seguenti tecnologie:
\begin{itemize}
    \item Visual Studio Code Extension API.
    \item Large Language Model(LLM) per l'analisi del codice e dei requisiti.
    \item Python o Node.js.
\end{itemize}

\subsubsection{Aspetti Positivi}
\begin{itemize}
    \item La proponente ha dichiarato un alta disponibilità.
    \item Integrazione interessante dei Large Language Model(LLM).
\end{itemize}

\subsubsection{Aspetti Critici}
\begin{itemize}
    \item Il gruppo non ritiene particolarmente interessante lo sviluppo di plug-in di Visual Studio Code.
    \item Presentazione ritenuta non molto convincente da parte del gruppo.
\end{itemize}



\subsection{C9 / BuddyBot}

\subsubsection{Descrizione}
\begin{itemize}
    \item \textbf{Proponente}: AzzurroDigitale
    \item \textbf{Committenti}: Tullio Vardanega e Riccardo Cardin
    \item \textbf{Obiettivo}: realizzare un assistente virtuale in grado di fornire informazioni sui progetti aziendali attivi usando un Large Langue Model(LLM) e i dati registrati in un insieme di servizi terzi usati per le comunicazioni aziendali e per la gestione dei progetti.
\end{itemize}

\subsubsection{Dominio Applicativo}
Il progetto unisce il campo del \textit{Knowledge Management} con l’applicazione dell’AI per permettere un accesso diretto e intuitivo alle informazioni critiche dei progetti facilitando la collaborazione e la condivisione di conoscenza all’interno del team con la conseguente diminuzione degli errori.
Il progetto mira quindi ad affrontare una necessità comune a tutte le aziende software.

\subsubsection{Dominio Tecnologico}
La proponente ha suggerito l’utilizzo delle seguenti tecnologie:
\begin{itemize}
    \item \textbf{OpenAI}: usato come motore per le funzionalità di Natural Language Processing NLP.
    \item \textbf{LangChain}: progetto open-source che consente di utilizzare modelli di IA come blackbox senza richiedere una conoscenza dettagliata dei loro funzionamenti interni.
    \item \textbf{NestJS}: framework che permette la creazione di applicazioni Node.js server-side scalabili ed efficienti.
    \item \textbf{Spring Boot}: framework java per la creazione di applicazioni server-side.
    \item \textbf{Angular}: framework front-end per lo sviluppo di applicazioni web dinamiche.
    
\end{itemize}

\subsubsection{Aspetti Positivi}
\begin{itemize}
    \item Il progetto prevede l’integrazione di Large Language Model(LLM) per risolvere un problema concreto e probabilmente comune a molte aziende software, cioè riuscire a reperire informazioni corrette e in tempo rapido riguardanti documentazione e codice aziendale.
    Il gruppo è rimasto piacevolmente sorpreso dall'obiettivo del progetto.
    
    \item La proponente ha offerto molto supporto ai gruppi che decideranno di eseguire il progetto.
    
    \item La proponente suggerisce l’utilizzo della metodologia \textit{agile} che è molto usata e quindi appetibile dal punto di vista curriculare.
    
    \item L’utilizzo di framework front-end e back-end molto diffusi rappresenta un valore aggiunto dato che la loro conoscenza è spendibile nel mondo del lavoro.
    
    \item La presentazione del capitolato è stata convincente e curata.
    
\end{itemize}

\subsubsection{Aspetti Critici}
\begin{itemize}
    \item La Proponente non è stata chiara sul onere dei costi relativi all’utilizzo del LLM consigliato(la questione è stata discussa nell'incontro con il gruppo).
\end{itemize}
\end{document}
