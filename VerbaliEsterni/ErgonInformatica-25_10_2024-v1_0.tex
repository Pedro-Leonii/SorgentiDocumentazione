\documentclass[a4paper, 12pt]{article}

\usepackage[italian]{babel}
\usepackage{tikz}
\usepackage{xcolor}
\usepackage{graphicx}
\usepackage{hyperref}
\usepackage{imakeidx}
\usepackage{caption}
\usepackage{fancyhdr}
\usepackage{tabularx}


%--------------------VARIABILI--------------------
\def\logo{../Immagini/logo.jpeg}
\def\ultima-versione{v1.0}
\def\titolo{Verbale esterno Ergon Informatica }
%------------------------------------------------

\usetikzlibrary{calc}
\definecolor{fp-blue}{HTML}{2885c8}
\definecolor{fp-red}{HTML}{ea5f64}
\makeindex[title=Indice]
\hypersetup{hidelinks}

\pagestyle{fancy}
\fancyhead[L]{}
\setlength{\headheight}{15pt}
\fancyhead[R]{\titolo - \ultima-versione}

\renewcommand{\familydefault}{\sfdefault}
\newcommand{\glossario}[1]{\fontfamily{lmr}\selectfont{\textit{#1\textsubscript{\small G}}}}

%--------------------INFORMAZIONI PRIMA PAGINA-------------------- 
\title{\Huge \textbf{\titolo}}
\author{\Large{Alt} \raisebox{0.3ex}{\normalsize  +} \Large{F4}}
\date{25 ottobre 2024}
%----------------------------------------------------------------

\begin{document}

\begin{titlepage}      
    \maketitle
    \thispagestyle{empty}  

    \begin{tikzpicture}[remember picture, overlay]
        \fill[fp-blue] 
        ($(current page.south west) + (0, 10)$) 
        -- ($(current page.center) + (0, -8)$)
        -- ($(current page.center) + (0, -15)$)
        -- (current page.south west);

        \fill[fp-red]
        ($(current page.south east) + (0, 10)$) 
        -- ($(current page.center) + (0, -8)$)
        -- ($(current page.center) + (0, -15)$)
        -- (current page.south east);

        \clip ($(current page.center) + (0, -8)$) circle (1cm) node 
        {\includegraphics[width=.25\textwidth]{\logo}};
        
    \end{tikzpicture}    
\end{titlepage}

\tableofcontents

\newpage

\begin{table}[!h]
    \centering
    \caption*{\textbf{\Large Registro Modifiche}}
    {\renewcommand{\arraystretch}{2}
    \begin{tabularx}{\textwidth}{| X | X | X | X |}
        \hline
            \textbf{\large Versione} & 
            \textbf{\large Data} & 
            \textbf{\large Autore/i} & 
            \textbf{\large Descrizione} \\ 
        \hline \hline
        \ultima-versione & 
            30 ottobre  2024 & 
            Enrico Bianchi & 
            Approvazione documento \\
        \hline 
            v0.2 & 
            29 ottobre  2024 & 
            Pedro Leoni & 
            Revisione \\
        \hline 
            v0.1 & 
            29 ottobre  2024 & 
            Guirong Lan & 
            Stesura verbale \\
        \hline 
    \end{tabularx}}
\end{table}

\newpage

\section{Registro presenze}
\begin{itemize}
    \item[] \textbf{Data}: 25 ottobre 2024
    \item[] \textbf{Ora inizio}:  15:00
    \item[] \textbf{Ora fine}: 15:10
    \item[] \textbf{Piattaforma}: Zoom	
\end{itemize}
\begin{table}[!h]
\centering
{\renewcommand{\arraystretch}{2}
\begin{tabularx}{\textwidth}{| X | X |}
    \hline
        \textbf{\large Componente} & 
        \textbf{\large Presenza} \\ 
    \hline 
    \hline
        Eghosa Matteo Igbinedion Osamwonyi&
        Presente \\
    \hline 
        Guirong Lan&
        Presente \\
    \hline 
        Enrico Bianchi&
        Presente \\
    \hline 
        Francesco Savio&
        Presente \\
    \hline 
        Marko Peric&
        Presente \\
    \hline 
        Pedro Leoni&
        Presente \\
    \hline 

\end{tabularx}}
\end{table}

\begin{table}[!h]
    \centering
    {\renewcommand{\arraystretch}{2}
    \begin{tabularx}{\textwidth}{| X | X |}
        \hline
            \textbf{\large Nome} & 
            \textbf{\large Ruolo} \\ 
        \hline 
        \hline
            Gianluca Carlesso&
            Rappresentante dell'azienda \\
        \hline 

    \end{tabularx}}
\end{table}

\newpage

\section{Domande}
Di seguito sono riportate le domande fatte dal gruppo durante l'incontro:
\begin{enumerate}
    \item Il Large Language Model(LLM) verrà eseguito in locale oppure l'azienda fornirà un ambiente di esecuzione?
    \item Tutti i modelli di Large Language Models(LLM) consigliati sono open source? I modelli citati rappresentano solo un consiglio o un vincolo?
    \item Qual è la vostra disponibilità per gli incontri?
    \item Quale strumento di comunicazione preferite? Zoom va bene per voi?
    \item In poche parole, in cosa consiste un database vettoriale?
    \item Qual'è, secondo voi, la parte più difficile da realizzare?
    \item Se ci trovassimo in difficoltà sulle diverse tecnologie e non riuscissimo ad avanzare, sareste disposti a fornirci supporto?
    \item E' necessario realizzare un'interfaccia che si adatti a dispositivi desktop e mobile?
\end{enumerate}

\section{Conclusioni}
In base alle risposte fornite dai rappresentanti, il gruppo è arrivato alle seguenti conclusioni:
\begin{itemize}
    \item L'azienda proponente è disponibile previa discussione a fornire un ambiente di esecuzione per il Large Language Model scelto.

    \item I modelli consigliati dall’azienda proponente sono tutti open source e spetta al gruppo decidere quali opzioni preferire tra quelle suggerite o esplorare ulteriori alternative disponibili online.
    
    \item L'azienda propone incontri virtuali su Zoom, generalmente organizzabili entro 1-2 giorni dalla richiesta. In alternativa, è possibile valutare la possibilità di un incontro di persona presso la loro sede.
    
    \item In caso di difficoltà nell'avanzare, l'azienda proponente è disponibile a supportarci ed esaminare insieme le problematiche riscontrate.
    
    \item La sfida principale consiste nell'estrazione dei dati semantici da fornire al modello LLM pre-addestrato per permettergli di rispondere alle domande poste.

    \item L’obiettivo principale è quello di realizzare un interfaccia grafica adatta ai dispositivi mobile, poiché la maggior parte degli utenti interagirebbero col sistema tramite essi. Tuttavia, è anche possibile sviluppare una web app che si adatti a tutte le tipologie di dispositivi.
\end{itemize}

\vfill
{\renewcommand{\arraystretch}{2}
\begin{tabular}{l p{5cm}}
    Data: &  \hrulefill \\
    Firma: & \hrulefill \\
\end{tabular}
}

\end{document}
