\documentclass[a4paper, 12pt]{article}

\usepackage[italian]{babel}
\usepackage{tikz}
\usepackage{xcolor}
\usepackage{graphicx}
\usepackage{hyperref}
\usepackage{imakeidx}
\usepackage{caption}
\usepackage{fancyhdr}
\usepackage{tabularx}


%--------------------VARIABILI--------------------
\def\logo{../Immagini/logo.jpeg}
\def\ultima-versione{v1.0}
\def\titolo{Verbale esterno AzzurroDigitale }
%------------------------------------------------

\usetikzlibrary{calc}
\definecolor{fp-blue}{HTML}{2885c8}
\definecolor{fp-red}{HTML}{ea5f64}
\makeindex[title=Indice]
\hypersetup{hidelinks}

\pagestyle{fancy}
\fancyhead[L]{}
\setlength{\headheight}{15pt}
\fancyhead[R]{\titolo - \ultima-versione}

\renewcommand{\familydefault}{\sfdefault}
\newcommand{\glossario}[1]{\fontfamily{lmr}\selectfont{\textit{#1\textsubscript{\small G}}}}

%--------------------INFORMAZIONI PRIMA PAGINA-------------------- 
\title{\Huge \textbf{\titolo}}
\author{\Large{Alt} \raisebox{0.3ex}{\normalsize  +} \Large{F4}}
\date{25 ottobre 2024}
%----------------------------------------------------------------

\begin{document}

\begin{titlepage}      
    \maketitle
    \thispagestyle{empty}  

    \begin{tikzpicture}[remember picture, overlay]
        \fill[fp-blue] 
        ($(current page.south west) + (0, 10)$) 
        -- ($(current page.center) + (0, -8)$)
        -- ($(current page.center) + (0, -15)$)
        -- (current page.south west);

        \fill[fp-red]
        ($(current page.south east) + (0, 10)$) 
        -- ($(current page.center) + (0, -8)$)
        -- ($(current page.center) + (0, -15)$)
        -- (current page.south east);

        \clip ($(current page.center) + (0, -8)$) circle (1cm) node 
        {\includegraphics[width=.25\textwidth]{\logo}};
        
    \end{tikzpicture}    
\end{titlepage}

\tableofcontents

\newpage

\begin{table}[!h]
    \centering
    \caption*{\textbf{\Large Registro Modifiche}}
    {\renewcommand{\arraystretch}{2}
    \begin{tabularx}{\textwidth}{| X | X | X | X |}
        \hline
            \textbf{\large Versione} & 
            \textbf{\large Data} & 
            \textbf{\large Autore/i} & 
            \textbf{\large Descrizione} \\ 
        \hline \hline
        \ultima-versione & 
            30 ottobre 2024 & 
            Enrico Bianchi & 
            Approvazione documento\\
        \hline
            v0.2 & 
            29 ottobre 2024 & 
            Guirong Lan & 
            Revisione \\
        \hline 
            v0.1 & 
            29 ottobre 2024 & 
            Pedro Leoni & 
            Stesura verbale \\
        \hline  
    \end{tabularx}}
\end{table}

\newpage

\section{Registro presenze}
\begin{itemize}
    \item[] \textbf{Data}: 25 ottobre 2024
    \item[] \textbf{Ora inizio}:  16:30
    \item[] \textbf{Ora fine}: 16:50
    \item[] \textbf{Piattaforma}: Google Meet	
\end{itemize}
\begin{table}[!h]
\centering
{\renewcommand{\arraystretch}{2}
\begin{tabularx}{\textwidth}{| X | X |}
    \hline
        \textbf{\large Componente} & 
        \textbf{\large Presenza} \\ 
    \hline 
    \hline
        Eghosa Matteo Igbinedion Osamwonyi&
        Presente \\
    \hline 
        Guirong Lan&
        Presente \\
    \hline 
        Enrico Bianchi&
        Presente \\
    \hline 
        Francesco Savio&
        Presente \\
    \hline 
        Marko Peric&
        Presente \\
    \hline 
        Pedro Leoni&
        Presente \\
    \hline 

\end{tabularx}}
\end{table}

\begin{table}[!h]
    \centering
    {\renewcommand{\arraystretch}{2}
    \begin{tabularx}{\textwidth}{| X | X |}
        \hline
            \textbf{\large Nome} & 
            \textbf{\large Ruolo} \\ 
        \hline 
        \hline
            Nicola Boscaro&
            Rappresentante dell'azienda \\
        \hline 
            Martina Daniele&
            Rappresentante dell'azienda \\
        \hline 
    
    \end{tabularx}}
\end{table}

\newpage

\section{Domande}
Di seguito vengono riportate le domande fatte dal gruppo:
\begin{enumerate}
    \item  AzzurroDigitale fornirà esempi di progetto per le diverse piattaforme(Jira, Confluence e GitHub) con cui il sistema dovrà interfacciarsi, o sarà responsabilità del gruppo crearli?
    \item Qual è, secondo voi, l’aspetto più complesso del progetto?
    \item L’utilizzo di OpenAI è vincolante oppure è possibile optare per modelli open source? Nel caso in cui si scelga di utilizzare OpenAI, l’azienda provvederà a coprire interamente o parzialmente i costi?
    \item Cos'è LangChain e quale è la sua utilità?
    \item La chat deve essere visualizzabile su dispositivi mobili e desktop?
    \item  Cosa si intende per "documentare le API di terze parti usate"?
    \item  Con quale frequenza si terranno incontri per ottenere riscontri dall'azienda sullo stato di avanzamento del progetto?
    \item  Avete una preferenza tra i due framework(Spring Boot e NestJS) consigliati nel capitolato?
\end{enumerate}

\section{Conclusioni}
Il gruppo dalle risposte date dai rappresentanti dell'azienda proponente ha tratto le seguenti conclusioni:

\begin{itemize}
    \item La creazione degli account nelle piattaforme terze e il loro popolamento con dati di esempio sono lasciati a cura del gruppo. Eventualmente, dopo la creazione del Proof of Concept(PoC), il contenuto delle piattaforme Jira e Confluence potrà essere discusso con l'azienda proponente per verificarne la verosimiglianza. Per quanto riguarda la piattaforma GitHub, il gruppo potrà utilizzare come esempio un repository pubblico open source. 

    \item La sfida principale consiste nel garantire che le domande formulate tramite la chat vengano riconosciute e tradotte in richieste corrette verso le tre piattaforme esterne. Questa è la parte centrale del progetto e ha una rilevanza maggiore rispetto all'aspetto dell'interfaccia grafica, la cui responsiveness viene comunque valutata positivamente.

    \item L'API di OpenAI è solo un esempio di servizio utilizzabile per la parte di progetto riguardante i Large Language Model(LLM). Il gruppo può scegliere anche altri modelli open source. Dato che l'API di OpenAI è a pagamento, nel caso in cui il gruppo scelga di utilizzarla, l'azienda proponente, previa discussione, può sostenere parte dei costi. 

    \item L'azienda proponente richiede una giustificazione delle chiamate alle API dei servizi di terze parti, con un link alla documentazione ufficiale. 

    \item L'azienda proponente prevede di organizzare incontri di revisione inizialmente con cadenza bisettimanale, per poi passare a incontri settimanali dopo le prime dieci settimane dall'avvio del progetto. Inoltre, si intende attivare un canale di comunicazione informale (come Discord o Telegram) per facilitare una comunicazione più rapida riguardo a eventuali problematiche.

    \item La scelta del framework e del linguaggio da utilizzare per lo sviluppo del backend è a discrezione del gruppo. I framework NestJs e Spring Boot sono stati suggeriti in quanto già utilizzati internamente dall'azienda, che può quindi offrire supporto. Si raccomanda comunque di optare per il framework e il linguaggio con cui il gruppo ha maggiore familiarità.
\end{itemize}

\vfill
{\renewcommand{\arraystretch}{2}
\begin{tabular}{l p{5cm}}
    Data: &  \hrulefill \\
    Firma: & \hrulefill \\
\end{tabular}
}
\end{document}
