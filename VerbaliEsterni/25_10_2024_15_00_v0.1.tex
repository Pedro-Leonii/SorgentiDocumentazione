\documentclass[a4paper, 12pt]{article}

\usepackage[italian]{babel}
\usepackage{tikz}
\usepackage{xcolor}
\usepackage{graphicx}
\usepackage{hyperref}
\usepackage{imakeidx}
\usepackage{caption}
\usepackage{fancyhdr}
\usepackage{tabularx}


%--------------------VARIABILI--------------------
\def\logo{../Immagini/logo.jpeg}
\def\ultima-versione{v0.1}
\def\titolo{Verbale esterno Ergon Informatica }
%------------------------------------------------

\usetikzlibrary{calc}
\definecolor{fp-blue}{HTML}{2885c8}
\definecolor{fp-red}{HTML}{ea5f64}
\makeindex[title=Indice]
\hypersetup{hidelinks}

\pagestyle{fancy}
\fancyhead[L]{}
\setlength{\headheight}{15pt}
\fancyhead[R]{\titolo - \ultima-versione}

\renewcommand{\familydefault}{\sfdefault}
\newcommand{\glossario}[1]{\fontfamily{lmr}\selectfont{\textit{#1\textsubscript{\small G}}}}

%--------------------INFORMAZIONI PRIMA PAGINA-------------------- 
\title{\Huge \textbf{\titolo}}
\author{\Large{Alt} \raisebox{0.3ex}{\normalsize  +} \Large{F4}}
\date{\today}
%----------------------------------------------------------------

\begin{document}

\begin{titlepage}      
    \maketitle
    \thispagestyle{empty}  

    \begin{tikzpicture}[remember picture, overlay]
        \fill[fp-blue] 
        ($(current page.south west) + (0, 10)$) 
        -- ($(current page.center) + (0, -8)$)
        -- ($(current page.center) + (0, -15)$)
        -- (current page.south west);

        \fill[fp-red]
        ($(current page.south east) + (0, 10)$) 
        -- ($(current page.center) + (0, -8)$)
        -- ($(current page.center) + (0, -15)$)
        -- (current page.south east);

        \clip ($(current page.center) + (0, -8)$) circle (1cm) node 
        {\includegraphics[width=.25\textwidth]{\logo}};
        
    \end{tikzpicture}    
\end{titlepage}

\tableofcontents

\newpage

\begin{table}[!h]
    \centering
    \caption*{\textbf{\Large Registro Modifiche}}
    {\renewcommand{\arraystretch}{2}
    \begin{tabularx}{\textwidth}{| X | X | X | X |}
        \hline
            \textbf{\large Versione} & 
            \textbf{\large Data} & 
            \textbf{\large Autore/i} & 
            \textbf{\large Descrizione} \\ 
        \hline \hline
            \ultima-versione & 
            29 ottobre  2024 & 
            Guirong Lan & 
            Stesura verbale \\
        \hline 
    \end{tabularx}}
\end{table}

\newpage

\section{Registro presenze}
\begin{itemize}
    \item[] \textbf{Data}: 25 ottobre 2024
    \item[] \textbf{Ora inizio}:  15:00
    \item[] \textbf{Ora fine}: 15:10
    \item[] \textbf{Piattaforma}: Zoom	
\end{itemize}
\begin{table}[!h]
\centering
{\renewcommand{\arraystretch}{2}
\begin{tabularx}{\textwidth}{| X | X |}
    \hline
        \textbf{\large Componente} & 
        \textbf{\large Presenza} \\ 
    \hline 
    \hline
        Eghosa Igbinedion&
        Presente \\
    \hline 
        Giovanni Lan&
        Presente \\
    \hline 
        Enrico Bianchi&
        Presente \\
    \hline 
        Francesco Savio&
        Presente \\
    \hline 
        Marko Peric&
        Presente \\
    \hline 
        Pedro Leoni&
        Presente \\
    \hline 

\end{tabularx}}
\end{table}

\begin{table}[!h]
    \centering
    {\renewcommand{\arraystretch}{2}
    \begin{tabularx}{\textwidth}{| X | X |}
        \hline
            \textbf{\large Nome} & 
            \textbf{\large Ruolo} \\ 
        \hline 
        \hline
            Gianluca Carlesso&
            Rappresentante dell'azienda \\
        \hline 

    \end{tabularx}}
\end{table}

\newpage

\section{Domande}
Di seguito sono riportate le domande fatte dal gruppo durante l'incontro, che ci ha chiarito i nostri dubbi iniziali:
\begin{enumerate}
    \item Tutti i modelli di Large Language Models (LLM) consigliati sono open source? perché alcuni vengono citati mentre altri no?
    \item Qual è la vostra disponibilità per incontri?
    \item Quale strumento di comunicazione preferite? Zoom va bene per voi?
    \item In poche parole, in cosa consiste un database vettoriale?
    \item Qual'è, secondo voi, la parte più difficile da realizzare?
    \item Se ci trovassimo in difficoltà e non riusciamo ad avanzare, potremmo ricevere aiuto da parte vostra?
    \item Dal punto di vista implementativo, dobbiamo realizzare un'interfaccia sia per desktop che per mobile, o è sufficiente una sola versione?
\end{enumerate}

\section{Conclusioni}
In base alle risposte fornite dai rappresentanti, il gruppo è arrivato alle seguenti conclusioni:
\begin{itemize}
    \item I modelli consigliati dall’azienda proponente sono open source e spetta al gruppo decidere quali opzioni preferire tra quelle suggerite o esplorare ulteriori alternative disponibili online.
    
    \item L’azienda proponente ha assicurato piena disponibilità per organizzare incontri, solitamente entro 1-2 giorni. Inoltre, abbiamo concordato di utilizzare Zoom come strumento di comunicazione; in alternativa, si potrebbe anche organizzare un incontro di persona presso l’azienda.
    
    \item In caso di difficoltà nell'avanzare, l'azienda proponente è disponibile a supportarci e a esaminare insieme le nostre problematiche. Inoltre, ci invita a inviare un’email descrivendo la situazione, in modo da poter mandare un tecnico con competenze specifiche per fornire assistenza
    
    \item La sfida principale consiste nell'estrazione dei dati semantici dal modello LLM pre-addestrato, mentre le altre componenti, come la comunicazione, l'interfaccia utente e i servizi web API, non rappresentano ostacoli per lo sviluppo del progetto

    \item L’obiettivo principale deve essere progettato per la versione mobile, poiché molti utenti accedono tramite dispositivi portatili. Tuttavia, è anche possibile sviluppare una web app, lasciando al gruppo la decisione finale. Comunque, la gestione del backend deve comunque avvenire tramite una web app.
\end{itemize}

\vfill
{\renewcommand{\arraystretch}{2}
\begin{tabular}{l p{5cm}}
    Data: &  \hrulefill \\
    Firma: & \hrulefill \\
\end{tabular}
}

\end{document}