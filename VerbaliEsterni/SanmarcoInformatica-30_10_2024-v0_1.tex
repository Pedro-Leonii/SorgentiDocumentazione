\documentclass[a4paper, 12pt]{article}

\usepackage[italian]{babel}
\usepackage{tikz}
\usepackage{xcolor}
\usepackage{graphicx}
\usepackage{hyperref}
\usepackage{imakeidx}
\usepackage{caption}
\usepackage{fancyhdr}
\usepackage{tabularx}


%--------------------VARIABILI--------------------
\def\logo{../Immagini/logo.jpeg}
\def\ultima-versione{v0.2}
\def\titolo{Verbale esterno Sanmarco Informatica }
%------------------------------------------------

\usetikzlibrary{calc}
\definecolor{fp-blue}{HTML}{2885c8}
\definecolor{fp-red}{HTML}{ea5f64}
\makeindex[title=Indice]
\hypersetup{hidelinks}

\pagestyle{fancy}
\fancyhead[L]{}
\setlength{\headheight}{15pt}
\fancyhead[R]{\titolo - \ultima-versione}

\renewcommand{\familydefault}{\sfdefault}
\newcommand{\glossario}[1]{\fontfamily{lmr}\selectfont{\textit{#1\textsubscript{\small G}}}}

%--------------------INFORMAZIONI PRIMA PAGINA-------------------- 
\title{\Huge \textbf{\titolo}}
\author{\Large{Alt} \raisebox{0.3ex}{\normalsize  +} \Large{F4}}
\date{30 ottobre 2024}
%----------------------------------------------------------------

\begin{document}

\begin{titlepage}      
    \maketitle
    \thispagestyle{empty}  

    \begin{tikzpicture}[remember picture, overlay]
        \fill[fp-blue] 
        ($(current page.south west) + (0, 10)$) 
        -- ($(current page.center) + (0, -8)$)
        -- ($(current page.center) + (0, -15)$)
        -- (current page.south west);

        \fill[fp-red]
        ($(current page.south east) + (0, 10)$) 
        -- ($(current page.center) + (0, -8)$)
        -- ($(current page.center) + (0, -15)$)
        -- (current page.south east);

        \clip ($(current page.center) + (0, -8)$) circle (1cm) node 
        {\includegraphics[width=.25\textwidth]{\logo}};
        
    \end{tikzpicture}    
\end{titlepage}

\tableofcontents

\newpage

\begin{table}[!h]
    \centering
    \caption*{\textbf{\Large Registro Modifiche}}
    {\renewcommand{\arraystretch}{2}
    \begin{tabularx}{\textwidth}{| X | X | X | X |}
        \hline
            \textbf{\large Versione} & 
            \textbf{\large Data} & 
            \textbf{\large Autore/i} & 
            \textbf{\large Descrizione} \\ 
        \hline \hline
            \ultima-versione & 
            31 ottobre 2024 & 
            Enrico Bianchi & 
            Revisione documento \\
        \hline 
            v0.1 & 
            30 ottobre 2024 & 
            Francesco Savio & 
            Stesura verbale \\
        \hline 
    \end{tabularx}}
\end{table}

\newpage

\section{Registro presenze}
\begin{itemize}
    \item[] \textbf{Data}: 30 ottobre 2024
    \item[] \textbf{Ora inizio}:  17:30
    \item[] \textbf{Ora fine}: 17:50
    \item[] \textbf{Piattaforma}: Google Meet	
\end{itemize}
\begin{table}[!h]
\centering
{\renewcommand{\arraystretch}{2}
\begin{tabularx}{\textwidth}{| X | X |}
    \hline
        \textbf{\large Componente} & 
        \textbf{\large Presenza} \\ 
    \hline 
    \hline
        Eghosa Igbinedion&
        Presente \\
    \hline 
        Guirong Lan&
        Presente \\
    \hline 
        Enrico Bianchi&
        Presente \\
    \hline 
        Francesco Savio&
        Presente \\
    \hline 
        Marko Peric&
        Presente \\
    \hline 
        Pedro Leoni&
        Presente \\
    \hline 

\end{tabularx}}
\end{table}

\begin{table}[!h]
    \centering
    {\renewcommand{\arraystretch}{2}
    \begin{tabularx}{\textwidth}{| X | X |}
        \hline
            \textbf{\large Nome} & 
            \textbf{\large Ruolo} \\ 
        \hline 
        \hline
            Alex Beggiato&
            Rappresentante dell'azienda \\
        \hline 
    
    \end{tabularx}}
\end{table}

\newpage

\section{Domande}
Di seguito vengono riportate le domande fatte dal gruppo:
\begin{enumerate}
    \item  Il database con i valori da rappresentare verrà fornito da Sanmarco? Se si, sarà già popolato?
    \item Qual è, secondo voi, l’aspetto più complesso del progetto?
    \item Quali sono le differenze tra le due librerie grafiche proposte?
    \item L’utilizzo delle librerie grafiche 3D richiede particolari abilità matematiche?
    \item Quali sono le disponibilità dell’azienda e quale sarà la frequenza dei meeting di aggiornamento?
    \item  Come dovrà essere il database per la memorizzazione dei dati che verranno rappresentati graficamente?
    \item  Consigliate l’utilizzo di angular o react?
\end{enumerate}

\section{Conclusioni}
Il gruppo dalle risposte date dai rappresentanti dell'azienda proponente ha tratto le seguenti conclusioni:
\begin{itemize}
    \item Il database non verrà fornito dall'azienda, sarà compito nostro crearlo ed inizialmente potrà essere popolato manualmente o leggendo i dati da un semplice file di testo e successivamente i dati verranno presi da API o da altre fonti esterne.

    \item La parte più difficile del progetto è l'uso della libreria grafica e le interazioni che sono previste dal capitolato.

    \item D3js è una libreria che nasce in modo specifico per rappresentare i grafici bidimensionali mentre threejs nasce come motore generico per gestire il 3D nel browser ma non specificatamente per realizzazione di grafici. Sarà compito nostro scegliere la libreria che riteniamo più opportuna per soddisfare i requisiti del progetto.

    \item Queste librerie non richiedono particolari abilità matematiche.

    \item I meeting verranno fissati a inizio progetto con una schedulazione di 2 o 3 settimane e avrà una durata di circa 30 minuti, nel caso di avanzamenti rapidi la frequenza dei meeting potrà variare in base alle esigenze del gruppo.

    \item L'organizzazione del database in una o più tabelle è a discrezione nostra.
    
    \item Sono entrambi validi, la scelta dipende da noi anche in base ad eventuali conoscenze pregresse. React solitamente viene usato in un contesto di applicazioni semplici come siti web mentre angular viene solitamente usato in contesti professionali.
\end{itemize}


\vfill
{\renewcommand{\arraystretch}{2}
\begin{tabular}{l p{5cm}}
    Data: &  \hrulefill \\
    Firma: & \hrulefill \\
\end{tabular}
}
\end{document}