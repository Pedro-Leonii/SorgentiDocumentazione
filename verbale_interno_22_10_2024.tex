\documentclass[a4paper, 12pt]{article}

\usepackage[italian]{babel}
\usepackage{tikz}
\usepackage{xcolor}
\usepackage{graphicx}
\usepackage{hyperref}
\usepackage{imakeidx}
\usepackage{caption}
\usepackage{fancyhdr}
\usepackage{tabularx}


%--------------------VARIABILI--------------------
\def\logo{logo.jpeg}
\def\ultima-versione{v1.0}
\def\titolo{Verbale interno}
\def\data{22/10/2024}
%------------------------------------------------

\usetikzlibrary{calc}
\definecolor{fp-blue}{HTML}{2885c8}
\definecolor{fp-red}{HTML}{ea5f64}
\makeindex[title=Indice]
\hypersetup{hidelinks}

\pagestyle{fancy}
\fancyhead[L]{}
\setlength{\headheight}{15pt}
\fancyhead[R]{\titolo - \ultima-versione}

\renewcommand{\familydefault}{\sfdefault}
\newcommand{\glossario}[1]{\fontfamily{lmr}\selectfont{\textit{#1\textsubscript{\small G}}}}

%--------------------INFORMAZIONI PRIMA PAGINA-------------------- 
\title{\Huge \textbf{\titolo}}
\author{\Large{Alt} \raisebox{0.3ex}{\normalsize  +} \Large{F4}}
\date{22/10/2024}
%----------------------------------------------------------------

\begin{document}

\begin{titlepage}      
    \maketitle
    \thispagestyle{empty}  

    \begin{tikzpicture}[remember picture, overlay]
        \fill[fp-blue] 
        ($(current page.south west) + (0, 10)$) 
        -- ($(current page.center) + (0, -8)$)
        -- ($(current page.center) + (0, -15)$)
        -- (current page.south west);

        \fill[fp-red]
        ($(current page.south east) + (0, 10)$) 
        -- ($(current page.center) + (0, -8)$)
        -- ($(current page.center) + (0, -15)$)
        -- (current page.south east);

        \clip ($(current page.center) + (0, -8)$) circle (1cm) node 
        {\includegraphics[width=.25\textwidth]{\logo}};
        
    \end{tikzpicture}    
\end{titlepage}

\tableofcontents

\newpage

\begin{table}[!h]
    \centering
    \caption*{\textbf{\Large Registro Modifiche}}
    {\renewcommand{\arraystretch}{2}
    \begin{tabularx}{\textwidth}{| X | X | X | X |}
        \hline
            \textbf{\large Versione} & 
            \textbf{\large Data} & 
            \textbf{\large Autore/i} & 
            \textbf{\large Descrizione} \\ 
        \hline \hline
            0.1 & 
            \data & 
            Eghosa Igbinedion & 
            Prima stesura del documento \\
        \hline 
    \end{tabularx}}
\end{table}

\newpage

\section{Registro presenze}
   \begin{itemize}
        \item[] \textbf{Ora inizio}:  18:00
        \item[] \textbf{Ora fine}: 19:30
        \item[] \textbf{Piattaforma}: Discord	
    \end{itemize}
\begin{table}[!h]
    \centering
    {\renewcommand{\arraystretch}{2}
    \begin{tabularx}{\textwidth}{| X | X |}
        \hline
            \textbf{\large Componente} & 
            \textbf{\large Presenza} \\ 
        \hline 
        \hline
            Eghosa Igbinedion&
            Presente \\
        \hline 
        \hline
            Giovanni Lan&
            Presente \\
        \hline 
        \hline
            Enrico&
            Presente \\
        \hline 
        \hline
            Francesco&
            Presente \\
        \hline 
        \hline
            Marko Peric&
            Presente \\
        \hline 
        \hline
            Pedro Leoni&
            Presente \\
        \hline 

    \end{tabularx}}
\end{table}

\newpage

\section{Strumetni di lavoro}
Si discute l’utilizzo degli strumenti principali per il progetto. Dopo un confronto, il gruppo decide di adottare i seguenti strumenti:
\begin{itemize}
    \item \textbf{GitHub}: per la gestione del codice e il versionamento.
    \item \textbf{LaTeX}: per la stesura di documenti formali.
    \item \textbf{Discord}: come piattaforma di comunicazione principale per le riunioni e lo scambio quotidiano di informazioni.
\end{itemize}

\section{Strumetni di lavoro}
Durante la riunione, il gruppo si concentra sulla preparazione di alcune domande per comprendere meglio la natura dei capitolati C5, C7 e C9, precedentemente selezionati come i più interessanti. Le domande si focalizzano su:
\begin{itemize}
    \item  Le specifiche tecniche richieste per ciascun capitolato.
    \item Le aspettative delle aziende in merito alla fase di sviluppo e alle tecnologie da utilizzare.
    \item La flessibilità nel modificare alcune funzionalità richieste nel corso del progetto.
    \item Il livello di interazione previsto tra il gruppo e l'azienda durante il progetto.
\end{itemize}

\section{Workflow e requisiti minimi}
 Il gruppo discute poi il workflow da adottare per il completamento dei requisiti minimi necessari alla candidatura a uno dei capitolati. Viene concordato di sfruttare il sistema di \textbf{issue di GitHub} per:
\begin{itemize}
    \item  Tracciare e assegnare compiti e responsabilità.
    \item Monitorare lo stato di avanzamento dei requisiti.
\end{itemize}
Ogni membro del gruppo sarà incaricato di aprire e gestire issue legate ai propri compiti, garantendo trasparenza e tracciabilità dei progressi.

\section{Conclusioni}
L’incontro si conclude con la conferma della necessità di ulteriori approfondimenti sui capitolati scelti e l’organizzazione di una prossima riunione per valutare le risposte ricevute dalle aziende.

% ESEMPIO ELENCO PUNTATO
%\begin{itemize}
%    \item Primo elemento.
%    \item Secondo elemento.
%\end{itemize}

% ESEMPIO ELENCO NUMERATO
%\begin{enumerate}
%    \item Primo elemento.
%    \item Secondo elemento.
%\end{enumerate}

%\glossario{Termine glossario}
\end{document}