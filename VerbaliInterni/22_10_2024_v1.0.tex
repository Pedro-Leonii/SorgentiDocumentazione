\documentclass[a4paper, 12pt]{article}

\usepackage[italian]{babel}
\usepackage{tikz}
\usepackage{xcolor}
\usepackage{graphicx}
\usepackage{hyperref}
\usepackage{imakeidx}
\usepackage{caption}
\usepackage{fancyhdr}
\usepackage{tabularx}

%--------------------VARIABILI-------------------- 
\def\logo{../Immagini/logo.jpeg}
\def\ultima-versione{v1.0}
\def\titolo{Verbale interno }
\def\data{\today}
%------------------------------------------------

\usetikzlibrary{calc}
\definecolor{fp-blue}{HTML}{2885c8}
\definecolor{fp-red}{HTML}{ea5f64}
\makeindex[title=Indice]
\hypersetup{hidelinks}

\pagestyle{fancy}
\fancyhead[L]{}
\setlength{\headheight}{15pt}
\fancyhead[R]{\titolo - \ultima-versione}

\renewcommand{\familydefault}{\sfdefault}
\newcommand{\glossario}[1]{\fontfamily{lmr}\selectfont{\textit{#1\textsubscript{\small G}}}}

%--------------------INFORMAZIONI PRIMA PAGINA-------------------- 
\title{\Huge \textbf{\titolo}}
\author{\Large{Alt} \raisebox{0.3ex}{\normalsize  +} \Large{F4}}
\date{\data}
%----------------------------------------------------------------

\begin{document}

\begin{titlepage}      
    \maketitle
    \thispagestyle{empty}  

    \begin{tikzpicture}[remember picture, overlay]
        \fill[fp-blue] 
        ($(current page.south west) + (0, 10)$) 
        -- ($(current page.center) + (0, -8)$)
        -- ($(current page.center) + (0, -15)$)
        -- (current page.south west);

        \fill[fp-red]
        ($(current page.south east) + (0, 10)$) 
        -- ($(current page.center) + (0, -8)$)
        -- ($(current page.center) + (0, -15)$)
        -- (current page.south east);

        \clip ($(current page.center) + (0, -8)$) circle (1cm) node 
        {\includegraphics[width=.25\textwidth]{\logo}};
        
    \end{tikzpicture}    
\end{titlepage}

\tableofcontents

\newpage

\begin{table}[!h]
    \centering
    \caption*{\textbf{\Large Registro Modifiche}}
    {\renewcommand{\arraystretch}{2}
    \begin{tabularx}{\textwidth}{| X | X | X | X |}
        \hline
            \textbf{\large Versione} & 
            \textbf{\large Data} & 
            \textbf{\large Autore/i} & 
            \textbf{\large Descrizione} \\ 
        \hline \hline
            v0.1 & 
            27 ottobre 2024 & 
            Eghosa Igbinedion & 
            Prima stesura del documento \\ 
        \hline 
           v0.2 & 
            27 ottobre 2024 & 
            Marko Peric & 
            Revisione e modifica documento \\
        \hline  
            \ultima-versione & 
            27 ottobre 2024 & 
            Pedro Leoni & 
            Revisione \\
        \hline  
    \end{tabularx}}
\end{table}

\newpage

\section{Registro presenze}
   \begin{itemize}
        \item[] \textbf{Data}: 22 ottobre 2024
        \item[] \textbf{Ora inizio}: 18:00
        \item[] \textbf{Ora fine}: 19:30
        \item[] \textbf{Piattaforma}: Discord	
    \end{itemize}
\begin{table}[!h]
    \centering
    {\renewcommand{\arraystretch}{2}
    \begin{tabularx}{\textwidth}{| X | X |}
        \hline
            \textbf{\large Componente} & 
            \textbf{\large Presenza} \\ 
        \hline 
            Eghosa Igbinedion &
            Presente \\ 
        \hline 
            Giovanni Lan &
            Presente \\ 
        \hline 
            Enrico Bianchi &
            Presente \\ 
        \hline 
            Francesco Savio &
            Presente \\ 
        \hline 
            Marko Peric &
            Presente \\ 
        \hline 
            Pedro Leoni &
            Presente \\ 
        \hline 
    \end{tabularx}}
\end{table}

\newpage

\section{Way of working}
Il gruppo ha discusso sul workflow da adottare per il completamento dei requisiti minimi necessari alla candidatura a uno dei capitolati.

\subsection{Template}
Grazie alla maggiore esperienza di Pedro Leoni con lo strumento LaTeX, che si è deciso di usare per la stesura dei documenti, quest'ultimo si è offerto di creare un documento di esempio da utilizzare come template per i successivi verbali del gruppo.

\subsection{Gitflow}
Si è deciso di utilizzare gitflow come struttura di sviluppo per garantire un flusso di lavoro ordinato e scalabile, particolarmente utile in un progetto di team. Questo approccio permette di ridurre i conflitti e mantenere un elevato controllo sul codice in ogni fase del processo di sviluppo. La struttura si basa su branch principali:

\subsubsection{Main}
Contiene solo il codice stabile e pronto per il rilascio, assicurando che le versioni finali siano sempre affidabili.

\subsubsection{Develop}
Dove le funzionalità approvate dai branch di feature vengono integrate e testate, permettendo uno sviluppo continuo senza compromettere la stabilità del codice.

\subsubsection{Feature}
Creato per ogni nuova funzionalità, consente agli sviluppatori di lavorare in modo indipendente senza interferire con altre parti del progetto. Una volta completata, la feature viene unita a develop, garantendo così che solo le modifiche pronte siano integrate.

\subsection{Organizzazione della repository}
Si è concordato di sfruttare il sistema di \textbf{Issues}, il sistema di \textbf{Milestone} e il sistema di \textbf{Tables} per:
\begin{itemize}
    \item Tracciare e assegnare compiti e responsabilità.
    \item Monitorare lo stato di avanzamento dei requisiti.
    \item Creare compiti non assegnati che possono essere svolti in caso di completamento anticipato dei propri compiti.
\end{itemize}
Ogni membro del gruppo sarà incaricato di aprire e gestire le issue legate ai propri compiti, garantendo trasparenza e tracciabilità dei progressi.

\section{Brainstorming per le riunioni esterne}
Durante la riunione, il gruppo si è focalizzato sulla preparazione di domande mirate a comprendere in maniera più approfondita la natura dei capitolati C5, C7 e C9, precedentemente selezionati come i più rilevanti. Le domande sono state suddivise in quesiti di carattere generale, comuni a tutte le aziende, e domande tecniche specifiche per ogni capitolato. Di seguito sono riportati alcuni esempi.

\subsection{Domande generali}
\begin{itemize}
    \item Quali canali di comunicazione si intendano utilizzare per mantenere il contatto tra azienda e gruppo di sviluppo;
    \item La frequenza di interazione prevista tra azienda e gruppo durante le diverse fasi del progetto;
    \item La piattaforma di destinazione per il rilascio e la fruizione del prodotto finale;
    \item Quali componenti o aspetti del progetto, secondo il parere degli sviluppatori aziendali, potrebbero presentare le maggiori difficoltà tecniche.
\end{itemize}

\subsection{Domande tecniche}
\subsubsection{Azzurro Digitale (C9)}
\begin{itemize}
    \item Chiarimenti in merito alla menzione della "documentazione di API di terze parti" nel capitolato;
    \item Si richiede chiarimenti riguardo alla disponibilità di esempi di progetto per le diverse piattaforme con cui il sistema dovrà interfacciarsi, oppure se la responsabilità per la loro creazione ricade sul gruppo;
    \item Preferenze specifiche riguardo all'uso di Node.js o Spring Boot per lo sviluppo del progetto;
    \item Precisazioni sulla funzione principale del framework \textit{Langchain} e sul suo ruolo previsto nel progetto.
\end{itemize}

\subsubsection{Ergon Informatica (C7)}
\begin{itemize}
    \item Chiarimenti riguardo alla necessità di sviluppare un ambiente per la gestione di Large Language Model (LLM) o se tale ambiente venga fornito dall'azienda;
    \item Dettagli sulla gestione e struttura dei Database Vettoriali richiesti dal capitolato.
\end{itemize}

\subsubsection{SanMarco Informatica (C5)}
\begin{itemize}
    \item Se per l’apprendimento delle librerie \textit{three.js} e \textit{d3.js} siano previste risorse aziendali o se tale apprendimento debba avvenire in modo autonomo; 
    \item Eventuali restrizioni o linee guida specifiche per lo sviluppo del \textit{backend} del progetto. 
\end{itemize}

\section{Conclusioni}
L’incontro si conclude con la conferma della necessità di ulteriori approfondimenti sui capitolati scelti e l’organizzazione di una prossima riunione per valutare le risposte ricevute dalle aziende.
\end{document}